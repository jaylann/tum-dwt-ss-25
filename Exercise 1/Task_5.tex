\documentclass[11pt,a4paper]{article}

% Preamble: Loading necessary packages
\usepackage[a4paper, margin=2.5cm]{geometry}
\usepackage{amsmath, amssymb, amsthm}
\usepackage{hyperref}
\usepackage{xcolor}

% Hyperlink setup for clickable references
\hypersetup{
    colorlinks=true,
    linkcolor=blue,
    filecolor=magenta,
    urlcolor=cyan,
    citecolor=green,
    pdftitle={Exercise Walkthrough: Borel Sigma-Algebra Generators},
}

% Author and Title Information
\title{Exercise Walkthrough: Borel $\sigma$-Algebra Generators}
\author{Justin Lanfermann}
\date{25. June 2025}

% Custom theorem style for definitions/lemmas
\newtheoremstyle{tutorstyle}
  {\topsep} % Space above
  {\topsep} % Space below
  {\itshape} % Body font
  {} % Indent amount
  {\bfseries} % Theorem head font
  {.} % Punctuation after theorem head
  {.5em} % Space after theorem head
  {} % Theorem head spec (can be left empty)
\theoremstyle{tutorstyle}
\newtheorem*{definition}{Definition}
\newtheorem*{lemma}{Lemma}

\begin{document}

\maketitle

\section*{Overview: The Goal and the Strategy}

Welcome! This exercise asks us to prove that the \textbf{Borel $\sigma$-algebra on $\mathbb{R}^n$}, which we'll denote as $\mathcal{B}^n$, can be generated by two different collections of sets:
\begin{enumerate}
    \item $E_1 := \{A \subseteq \mathbb{R}^n : A \text{ is open}\}$, the collection of all open sets.
    \item $E_2 := \{[a, b) : a, b \in \mathbb{Q}^n, a < b\}$, the collection of all half-open intervals with rational endpoints.
\end{enumerate}
In the language of the script, we need to show that $\sigma(E_1) = \sigma(E_2)$.

\paragraph{Our Strategy:}
To prove that two sets (in our case, two $\sigma$-algebras) are equal, the standard method is to show mutual inclusion. That is, we will prove:
\begin{itemize}
    \item \textbf{Part 1:} $\sigma(E_1) \subseteq \sigma(E_2)$
    \item \textbf{Part 2:} $\sigma(E_2) \subseteq \sigma(E_1)$
\end{itemize}
To do this, we'll use a key property of generated $\sigma$-algebras from \textbf{Lemma 1.7}\hyperlink{concept:generator}{[1]}: $\sigma(E)$ is the \textit{smallest} $\sigma$-algebra containing the generating set $E$. This means if we can show that the generating set $E_i$ is a subset of the $\sigma$-algebra $\sigma(E_j)$, it automatically follows that $\sigma(E_i) \subseteq \sigma(E_j)$.

So, our proof boils down to showing:
\begin{itemize}
    \item \textbf{Part 1:} $E_1 \subseteq \sigma(E_2)$ (i.e., every open set can be constructed from half-open rational intervals).
    \item \textbf{Part 2:} $E_2 \subseteq \sigma(E_1)$ (i.e., every half-open rational interval can be constructed from open sets).
\end{itemize}

Let's tackle each part.

\section*{Part 1: Showing $\sigma(E_1) \subseteq \sigma(E_2)$}

\paragraph{The Goal:} We need to show that any open set $A \in E_1$ is also an element of $\sigma(E_2)$.

\paragraph{The Reasoning:}
The definition of a $\sigma$-algebra \hyperlink{concept:sigma-algebra}{[2]} tells us it is closed under \textit{countable} unions. The key idea here is to represent any open set $A$ as a \textit{countable union} of sets from our generator $E_2$.

Why is this possible? Because the set of rational numbers $\mathbb{Q}$ is dense in the real numbers $\mathbb{R}$. This property extends to $\mathbb{Q}^n$ being dense in $\mathbb{R}^n$ \hyperlink{concept:dense}{[3]}. This means that for any point in $\mathbb{R}^n$, we can find a point in $\mathbb{Q}^n$ that is arbitrarily close. We can leverage this to "fill" any open set with a countable number of our rational intervals.

\paragraph{The Proof Steps:}
\begin{enumerate}
    \item \textbf{Take an arbitrary open set.} Let $A \subseteq \mathbb{R}^n$ be any open set, so $A \in E_1$.

    \item \textbf{Use the definition of an open set.} For any point $x \in A$, by definition of an open set, there exists an open ball $B_\epsilon(x)$ with radius $\epsilon > 0$ such that $x \in B_\epsilon(x) \subseteq A$.

    \item \textbf{Find a rational interval inside the ball.} Because $\mathbb{Q}^n$ is dense in $\mathbb{R}^n$, we can find a half-open interval $I_x = [a, b)$ with rational endpoints $a, b \in \mathbb{Q}^n$ such that $x \in I_x \subseteq B_\epsilon(x)$. Since $B_\epsilon(x) \subseteq A$, we have $x \in I_x \subseteq A$.

    \textit{Justification:} We can always find such an interval. For instance, we can find rational coordinates $a_i < x_i$ and $b_i > x_i$ for each dimension $i$, and make them close enough to $x_i$ so that the resulting box $[a,b)$ lies entirely within the ball $B_\epsilon(x)$.

    \item \textbf{Construct the open set as a countable union.} Let $\mathcal{C}$ be the collection of all possible half-open intervals with rational endpoints: $\mathcal{C} = \{[a,b) : a,b \in \mathbb{Q}^n, a<b\}$. This collection $\mathcal{C}$ is countable \hyperlink{concept:countable}{[4]} because it's indexed by pairs of elements from the countable set $\mathbb{Q}^n$.

    Now, for our open set $A$, we can write it as the union of all the rational intervals from $\mathcal{C}$ that are entirely contained within $A$:
    \[ A = \bigcup_{\{I \in \mathcal{C} \,|\, I \subseteq A\}} I \]

    \textit{Justification:} The inclusion $\bigcup \subseteq A$ is true by definition. For the other direction ($A \subseteq \bigcup$), we showed in step 3 that for any $x \in A$, there is at least one such interval $I_x$ in our collection that contains $x$. Therefore, every point in $A$ is included in the union.

    \item \textbf{Conclusion for Part 1.} We have just shown that any open set $A$ can be written as a countable union of sets of the form $[a,b)$ where $a,b \in \mathbb{Q}^n$. Since each of these intervals is in our generating set $E_2$, and $\sigma(E_2)$ is closed under countable unions (by \textbf{Definition 1.5}), it must be that $A \in \sigma(E_2)$.
\end{enumerate}

Since we have shown that any arbitrary set in $E_1$ is also in $\sigma(E_2)$, we have $E_1 \subseteq \sigma(E_2)$. By the property of the generated $\sigma$-algebra being the smallest one containing its generator, we conclude:
\[ \sigma(E_1) \subseteq \sigma(E_2) \]

\section*{Part 2: Showing $\sigma(E_2) \subseteq \sigma(E_1)$}

\paragraph{The Goal:} We need to show that any half-open rational interval $[a,b) \in E_2$ is also an element of $\sigma(E_1)$.

\paragraph{The Reasoning:}
This part is more direct. We need to construct the set $[a,b)$ using operations on open sets. The trick is to represent the "closed" part of the interval boundary (the `$[$` at $a$) using a countable intersection of open intervals that "shrink" towards it.

\paragraph{The Proof Steps:}
\begin{enumerate}
    \item \textbf{Take an arbitrary rational interval.} Let $[a, b)$ be any set in $E_2$, where $a, b \in \mathbb{Q}^n$ and $a < b$.

    \item \textbf{Construct it from open sets.} Consider the following representation:
    \[ [a, b) = \bigcap_{k=1}^{\infty} \left( a - \frac{\mathbf{1}}{k}, b \right) \]
    where $\mathbf{1} = (1, 1, \dots, 1) \in \mathbb{R}^n$ is the vector of all ones, and the inequalities and operations are component-wise.

    \item \textbf{Analyze the construction.}
    \begin{itemize}
        \item For any integer $k \ge 1$, the set $O_k := \left( a - \frac{\mathbf{1}}{k}, b \right)$ is an open rectangle (or hyperrectangle). Every open rectangle is an open set in $\mathbb{R}^n$. Therefore, $O_k \in E_1$ for all $k$.
        \item Our construction expresses $[a,b)$ as a \textit{countable intersection} of these open sets $O_k$.
        \item Since $\sigma(E_1)$ is a $\sigma$-algebra, it is closed under countable intersections (this follows from being closed under complements and countable unions, see \textbf{Corollary 1.9}).
    \end{itemize}

    \textit{Justification of the set equality:}
    \begin{itemize}
        \item ($\subseteq$): Let $x \in [a, b)$. This means $a_i \le x_i < b_i$ for all components $i$. For any $k \ge 1$, we have $a_i - 1/k < a_i$, so it's clear that $a_i - 1/k < x_i < b_i$. Thus, $x \in (a - \mathbf{1}/k, b)$ for all $k$, which means $x \in \bigcap_{k=1}^{\infty} O_k$.
        \item ($\supseteq$): Let $x \in \bigcap_{k=1}^{\infty} O_k$. This means that for every $k \ge 1$, we have $a_i - 1/k < x_i < b_i$ for all $i$. The condition $x_i < b_i$ is immediate. The other condition, $a_i - 1/k < x_i$, holds for all $k$. If we take the limit as $k \to \infty$, we get $a_i \le x_i$. Combining these gives $a \le x < b$, so $x \in [a, b)$.
    \end{itemize}

    \item \textbf{Conclusion for Part 2.} We have successfully written $[a,b)$ as a countable intersection of open sets. Since each of these open sets is in $\sigma(E_1)$ (as they are in $E_1$ itself), their countable intersection must also be in $\sigma(E_1)$.
\end{enumerate}

Therefore, $E_2 \subseteq \sigma(E_1)$, which implies:
\[ \sigma(E_2) \subseteq \sigma(E_1) \]

\section*{Final Conclusion and Summary}

We have successfully shown both $\sigma(E_1) \subseteq \sigma(E_2)$ and $\sigma(E_2) \subseteq \sigma(E_1)$. This allows us to conclude that the two $\sigma$-algebras are indeed identical:
\[ \mathcal{B}^n = \sigma(E_1) = \sigma(E_2) \]

\paragraph{Key Takeaways:}
\begin{itemize}
    \item The proof hinges on the standard strategy of proving mutual set inclusion ($A=B \iff A \subseteq B \land B \subseteq A$).
    \item \textbf{Part 1 (open sets from rational intervals):} The crucial trick was using the \textbf{density of $\mathbb{Q}^n$ in $\mathbb{R}^n$} to build any open set from a \textbf{countable union} of our generator sets.
    \item \textbf{Part 2 (rational intervals from open sets):} The key was to use a \textbf{countable intersection} of expanding open intervals to construct the "closed" end of our half-open interval.
\end{itemize}

This result is powerful. It means that to prove something holds for all Borel sets, we often only need to prove it for a much simpler class of sets (like half-open rational intervals) and then use extension theorems.

\subsection*{Check Your Understanding}
The solution remarks that the collection of \textbf{closed sets}, $E_5 := \{A \subseteq \mathbb{R}^n : A \text{ is closed}\}$, also generates $\mathcal{B}^n$. Can you sketch a quick argument for why $\sigma(E_1) = \sigma(E_5)$?
\textit{Hint: What is the relationship between open and closed sets? How are $\sigma$-algebras defined with respect to that operation?}

\subsection*{Further Reading}
A natural follow-up question is: why did we use intervals with \textit{rational} endpoints? Could we use real endpoints? The answer is yes, but using rational endpoints gives us a \textit{countable} generating set, which is often mathematically convenient. The resulting $\sigma$-algebra is the same.

\newpage
\section*{In-depth Explanations}
\begin{enumerate}
    \item \label{concept:generator} \textbf{Generator of a $\sigma$-algebra (Lemma 1.7):} For any collection of subsets $E$ of a sample space $\Omega$, the $\sigma$-algebra generated by $E$, denoted $\sigma(E)$, is defined as the intersection of all possible $\sigma$-algebras on $\Omega$ that contain $E$. This makes it the \textit{smallest} $\sigma$-algebra that contains every set in $E$. This "smallest" property is the key to our proof strategy.

    \item \label{concept:sigma-algebra} \textbf{$\sigma$-algebra (Definition 1.5):} A collection of subsets $\mathcal{A}$ of a sample space $\Omega$ is a $\sigma$-algebra if it satisfies three properties:
        \begin{itemize}
            \item[(i)] It contains the empty set: $\emptyset \in \mathcal{A}$. (This implies $\Omega \in \mathcal{A}$ too).
            \item[(ii)] It is closed under complementation: If $A \in \mathcal{A}$, then its complement $A^c = \Omega \setminus A$ is also in $\mathcal{A}$.
            \item[(iii)] It is closed under countable unions: If $A_1, A_2, \dots$ is a countable sequence of sets in $\mathcal{A}$, then their union $\bigcup_{i=1}^\infty A_i$ is also in $\mathcal{A}$.
        \end{itemize}
    These properties ensure we can perform all the standard operations of probability theory on the sets within the $\sigma$-algebra.

    \item \label{concept:dense} \textbf{Density of $\mathbb{Q}^n$ in $\mathbb{R}^n$:} A set $Q$ is dense in a set $R$ if every point in $R$ is either a point in $Q$ or a "limit point" of $Q$. In simpler terms, for any point $x \in \mathbb{R}^n$ and any distance $\epsilon > 0$, you can always find a point $q \in \mathbb{Q}^n$ such that the distance between $x$ and $q$ is less than $\epsilon$. This is why we can "approximate" any location in real space with a rational point, allowing us to fill open sets with rational building blocks.

    \item \label{concept:countable} \textbf{Countable Sets:} A set is countable if its elements can be put into a one-to-one correspondence with the natural numbers $\{1, 2, 3, \dots\}$. The set of rational numbers $\mathbb{Q}$ is countable. The Cartesian product of a finite number of countable sets is also countable, which means $\mathbb{Q}^n$ is countable. The set $\mathcal{C}$ of rational intervals is indexed by pairs of elements from $\mathbb{Q}^n$, making it countable as well. The real numbers $\mathbb{R}$, however, are uncountable.
\end{enumerate}

\end{document}
