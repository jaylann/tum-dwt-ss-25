\documentclass[11pt,a4paper]{article}

% --- PACKAGES ---
\usepackage[a4paper, margin=2.5cm]{geometry}
\usepackage{amsmath}
\usepackage{amssymb}
\usepackage{amsfonts}
\usepackage{graphicx}
\usepackage[utf8]{inputenc}
\usepackage[T1]{fontenc}
\usepackage{hyperref}

% --- HYPERREF SETUP ---
\hypersetup{
    colorlinks=true,
    linkcolor=blue,
    urlcolor=blue,
    citecolor=red
}

% --- DOCUMENT INFORMATION ---
\title{\textbf{Exercise Walkthrough: Value at Risk Analysis}}
\author{Justin Lanfermann}
\date{June 25, 2025}

% --- CUSTOM COMMAND FOR CONCEPT NOTES ---
% This creates a clickable link in the text to a more detailed explanation at the end.
\newcommand{\concept}[2]{\hyperlink{note:#1}{#2}}

\begin{document}

\maketitle

\begin{abstract}
    This document provides a detailed, step-by-step walkthrough for an exercise on continuous random variables from the Discrete Probability Theory course. We will analyze the distribution of a stock's daily return, focusing on calculating its Cumulative Distribution Function (CDF), key statistical measures like the mean, median, and mode, and an important financial risk metric, the Value at Risk (VaR). Each step is accompanied by explanations of the underlying theory and the reasoning behind the calculations. Key concepts are linked to an appendix for deeper review.
\end{abstract}

\section{Problem Statement}

An investment analyst models the daily percentage return $R$ of a particular stock as a continuous random variable with the probability density function (PDF):
\[
p_R(r) =
\begin{cases}
    \frac{2}{15}(r+3) & \text{if } -3 \le r < 0 \\
    \frac{1}{5}(2-r) & \text{if } 0 \le r \le 2 \\
    0 & \text{otherwise.}
\end{cases}
\]
The return $r$ is given in percent (e.g., $r=1$ means a 1\% return). The Value at Risk (VaR) at a confidence level $\alpha$ is the maximum percentage loss $L$ such that the probability of experiencing a loss greater than $L$ is $\alpha$. Mathematically, if $L > 0$ is the VaR, then $P(R \le -L) = \alpha$.

\begin{enumerate}
    \item[(i)] Verify that $p_R(r)$ is a valid PDF.
    \item[(ii)] Calculate the cumulative distribution function (CDF), $F_R(r)$.
    \item[(iii)] Explain how the $(100\alpha)\%$ VaR relates to the $\alpha$-quantile of the return distribution.
    \item[(iv)] Determine the 5\% Value at Risk (VaR) for this investment.
    \item[(v)] Calculate the median, expected daily return, and the mode. Explain your findings.
    \item[(vi)] If an investor has \$250,000 in this stock, what is the dollar amount corresponding to the 5\% daily VaR?
\end{enumerate}

\section{Solution Walkthrough}

\subsection{Part (i): Verification of the PDF}
\paragraph{Reasoning:}
For a function to be a valid \concept{pdf}{Probability Density Function (PDF)} [1], it must satisfy two conditions based on \textit{Definition 1.39}:
\begin{enumerate}
    \item \textbf{Non-negativity:} The function must be greater than or equal to zero for all possible values, i.e., $p_R(r) \ge 0$ for all $r \in \mathbb{R}$.
    \item \textbf{Normalization:} The total area under the function's curve must be equal to 1. This means the integral of the PDF over its entire domain must be 1, i.e., $\int_{-\infty}^{\infty} p_R(r) \,dr = 1$.
\end{enumerate}
Let's check these two properties for the given $p_R(r)$.

\paragraph{Step 1: Check Non-negativity}
\begin{itemize}
    \item For $-3 \le r < 0$: The term $(r+3)$ ranges from $0$ to $3$. Thus, $\frac{2}{15}(r+3)$ is non-negative in this interval.
    \item For $0 \le r \le 2$: The term $(2-r)$ ranges from $2$ to $0$. Thus, $\frac{1}{5}(2-r)$ is non-negative in this interval.
    \item Elsewhere, the function is $0$.
\end{itemize}
The non-negativity condition holds for all $r \in \mathbb{R}$.

\paragraph{Step 2: Check Normalization}
We need to compute the integral of $p_R(r)$ from $-\infty$ to $\infty$. Since the PDF is piecewise, we split the integral into the defined segments.
\begin{align*}
    \int_{-\infty}^{\infty} p_R(r) \,dr &= \int_{-3}^{0} \frac{2}{15}(r+3) \,dr + \int_{0}^{2} \frac{1}{5}(2-r) \,dr \\
    &= \frac{2}{15} \left[ \frac{r^2}{2} + 3r \right]_{-3}^{0} + \frac{1}{5} \left[ 2r - \frac{r^2}{2} \right]_{0}^{2} \\
    &= \frac{2}{15} \left( (0) - \left(\frac{(-3)^2}{2} + 3(-3)\right) \right) + \frac{1}{5} \left( \left(2(2) - \frac{2^2}{2}\right) - (0) \right) \\
    &= \frac{2}{15} \left( - \left(\frac{9}{2} - 9\right) \right) + \frac{1}{5} \left( 4 - 2 \right) \\
    &= \frac{2}{15} \left( - \left(-\frac{9}{2}\right) \right) + \frac{1}{5} (2) \\
    &= \frac{2}{15} \left(\frac{9}{2}\right) + \frac{2}{5} \\
    &= \frac{9}{15} + \frac{2}{5} = \frac{3}{5} + \frac{2}{5} = 1
\end{align*}
The normalization condition holds. Since both conditions are satisfied, $p_R(r)$ is a valid PDF.

\subsection{Part (ii): Cumulative Distribution Function (CDF)}
\paragraph{Reasoning:}
The \concept{cdf}{Cumulative Distribution Function (CDF)} [2], denoted $F_R(r)$, gives the probability that the random variable $R$ takes a value less than or equal to $r$. It is calculated by integrating the PDF from $-\infty$ to $r$, as per \textit{Definition 1.21}:
\[ F_R(r) = P(R \le r) = \int_{-\infty}^{r} p_R(t) \,dt \]
We must evaluate this for all $r \in \mathbb{R}$, which again requires a piecewise approach.

\paragraph{Calculation:}
\begin{itemize}
    \item \textbf{For $r < -3$:} The PDF is 0, so the integral is 0. $F_R(r) = 0$.
    \item \textbf{For $-3 \le r < 0$:}
    \begin{align*}
        F_R(r) &= \int_{-\infty}^{r} p_R(t) \,dt = \int_{-3}^{r} \frac{2}{15}(t+3) \,dt = \frac{2}{15} \left[ \frac{t^2}{2} + 3t \right]_{-3}^{r} \\
        &= \frac{2}{15} \left( \left(\frac{r^2}{2} + 3r\right) - \left(\frac{9}{2} - 9\right) \right) = \frac{2}{15} \left( \frac{r^2}{2} + 3r + \frac{9}{2} \right) \\
        &= \frac{1}{15} (r^2 + 6r + 9) = \frac{(r+3)^2}{15}
    \end{align*}
    \item \textbf{For $0 \le r \le 2$:} The CDF is cumulative. We add the probability up to $r=0$ to the integral from $0$ to $r$.
    \[ F_R(r) = F_R(0) + \int_{0}^{r} p_R(t) \,dt \]
    First, find $F_R(0)$ using the formula for the previous interval: $F_R(0) = \frac{(0+3)^2}{15} = \frac{9}{15} = 0.6$.
    \begin{align*}
        F_R(r) &= 0.6 + \int_{0}^{r} \frac{1}{5}(2-t) \,dt = 0.6 + \frac{1}{5} \left[ 2t - \frac{t^2}{2} \right]_{0}^{r} \\
        &= \frac{6}{10} + \frac{1}{5} \left( 2r - \frac{r^2}{2} \right) = \frac{6}{10} + \frac{4r - r^2}{10} = \frac{6+4r-r^2}{10}
    \end{align*}
    \item \textbf{For $r \ge 2$:} The CDF at this point should be 1, as it covers all possible outcomes. Let's check $F_R(2)$:
    \[ F_R(2) = \frac{6+4(2)-2^2}{10} = \frac{6+8-4}{10} = \frac{10}{10} = 1 \]
    So, for any $r \ge 2$, $F_R(r)=1$.
\end{itemize}
\paragraph{Final CDF:}
Combining all pieces, the CDF is:
\[
F_R(r) =
\begin{cases}
    0 & \text{if } r < -3 \\
    \frac{(r+3)^2}{15} & \text{if } -3 \le r < 0 \\
    \frac{6+4r-r^2}{10} & \text{if } 0 \le r \le 2 \\
    1 & \text{if } r \ge 2
\end{cases}
\]

\subsection{Part (iii): Relation between VaR and Quantiles}
\paragraph{Reasoning:}
This part asks us to connect two important concepts. Let's define them carefully based on our script.
\begin{itemize}
    \item The \concept{var}{\textbf{Value at Risk (VaR)}} [4] at a confidence level $\alpha$ is a \textit{positive} loss amount $L$ such that the probability of the return $R$ being less than or equal to the negative of this loss (i.e., $R \le -L$) is $\alpha$.
    \[ P(R \le -L) = \alpha \]
    \item The \concept{quantile}{\textbf{$\alpha$-quantile}} [3] of a distribution, let's call it $r_\alpha$, is the value such that the probability of the random variable being less than or equal to it is $\alpha$. Per \textit{Definition 2.20}, this is:
    \[ P(R \le r_\alpha) = \alpha \quad \text{or equivalently} \quad F_R(r_\alpha) = \alpha \]
\end{itemize}
By directly comparing the two definitions, we can see that the return value corresponding to the VaR event, which is $-L$, is precisely the $\alpha$-quantile, $r_\alpha$.
\[ -L = r_\alpha \]
Therefore, the VaR, $L$, is the negative of the $\alpha$-quantile. Since the quantile is found by inverting the CDF, $r_\alpha = F_R^{-1}(\alpha)$, we have:
\[ \text{VaR} = L = -r_\alpha = -F_R^{-1}(\alpha) \]
This relationship is crucial: to find the VaR, we find the corresponding quantile of the return distribution and take its negative.

\subsection{Part (iv): 5\% Value at Risk (VaR)}
\paragraph{Reasoning:}
We apply the result from part (iii) with $\alpha = 0.05$. We need to find the loss $L$ such that $P(R \le -L) = 0.05$. This means we must first find the $0.05$-quantile, $r_{0.05}$, by solving $F_R(r_{0.05}) = 0.05$.

\paragraph{Step 1: Identify the correct CDF interval}
We need to know which piece of our piecewise CDF to use. We check the value of the CDF at the boundary point $r=0$: $F_R(0) = 0.6$.
Since $0.05 < 0.6$, the value $r_{0.05}$ that gives a CDF value of $0.05$ must lie in the first active interval, i.e., $-3 \le r_{0.05} < 0$.

\paragraph{Step 2: Solve for the quantile}
We use the CDF formula for this interval and set it equal to $0.05$:
\begin{align*}
    \frac{(r_{0.05}+3)^2}{15} &= 0.05 \\
    (r_{0.05}+3)^2 &= 0.05 \times 15 = 0.75 \\
    r_{0.05}+3 &= \pm\sqrt{0.75} = \pm\sqrt{\frac{3}{4}} = \pm\frac{\sqrt{3}}{2} \\
    r_{0.05} &= -3 \pm \frac{\sqrt{3}}{2}
\end{align*}
This gives two potential solutions:
$r_1 = -3 + \frac{\sqrt{3}}{2} \approx -3 + 0.866 = -2.134$
$r_2 = -3 - \frac{\sqrt{3}}{2} \approx -3 - 0.866 = -3.866$
Since we established that $r_{0.05}$ must be in the interval $[-3, 0)$, we must choose $r_1$. So, $r_{0.05} \approx -2.134$.

\paragraph{Step 3: Calculate VaR}
The 5\% VaR is the positive loss $L = -r_{0.05}$.
\[ L = - \left(-3 + \frac{\sqrt{3}}{2}\right) = 3 - \frac{\sqrt{3}}{2} \approx 2.134 \]
The 5\% Value at Risk is a loss of approximately \textbf{2.134\%}.

\subsection{Part (v): Median, Mean, and Mode}
\paragraph{Reasoning:}
We are asked to calculate three different measures of central tendency for the distribution.
\begin{itemize}
    \item \textbf{\concept{median}{Median}} [6]: The 50\% quantile ($r_{0.5}$), which splits the distribution's area in half. We solve $F_R(r_{0.5}) = 0.5$.
    \item \textbf{\concept{mean}{Expected Value (Mean)}} [5]: The "center of mass" of the distribution, calculated as $E[R] = \int_{-\infty}^{\infty} r \cdot p_R(r) \,dr$.
    \item \textbf{\concept{mode}{Mode}} [7]: The most likely outcome, which is the value of $r$ that maximizes the PDF, $p_R(r)$.
\end{itemize}

\paragraph{Calculations:}
\begin{itemize}
    \item \textbf{Median:} We solve $F_R(r_{0.5}) = 0.5$. Since $0.5 < F_R(0)=0.6$, the median must also lie in the interval $[-3, 0)$.
    \begin{align*}
        \frac{(r_{0.5}+3)^2}{15} &= 0.5 \\
        (r_{0.5}+3)^2 &= 7.5 \\
        r_{0.5}+3 &= \pm\sqrt{7.5} \\
        r_{0.5} &= -3 + \sqrt{7.5} \quad (\text{choosing the root in } [-3, 0)) \\
        &\approx -3 + 2.739 = -0.261
    \end{align*}
    The median return is approximately \textbf{-0.261\%}.

    \item \textbf{Expected Value (Mean):} We integrate $r \cdot p_R(r)$.
    \begin{align*}
        E[R] &= \int_{-3}^{0} r \left(\frac{2}{15}(r+3)\right) \,dr + \int_{0}^{2} r \left(\frac{1}{5}(2-r)\right) \,dr \\
        &= \frac{2}{15}\int_{-3}^{0} (r^2+3r) \,dr + \frac{1}{5}\int_{0}^{2} (2r-r^2) \,dr \\
        &= \frac{2}{15}\left[\frac{r^3}{3} + \frac{3r^2}{2}\right]_{-3}^{0} + \frac{1}{5}\left[r^2 - \frac{r^3}{3}\right]_{0}^{2} \\
        &= \frac{2}{15}\left(0 - \left(\frac{-27}{3} + \frac{27}{2}\right)\right) + \frac{1}{5}\left(\left(4 - \frac{8}{3}\right) - 0\right) \\
        &= \frac{2}{15}\left(- \left(-9 + 13.5\right)\right) + \frac{1}{5}\left(\frac{4}{3}\right) \\
        &= \frac{2}{15}(-4.5) + \frac{4}{15} = -\frac{9}{15} + \frac{4}{15} = -\frac{5}{15} = -\frac{1}{3}
    \end{align*}
    The expected return is \textbf{-0.333\%}.

    \item \textbf{Mode:} We find the maximum of the PDF $p_R(r)$. The PDF consists of two linear segments: an increasing line from $r=-3$ to $r=0$, and a decreasing line from $r=0$ to $r=2$. The function's peak is at the point where they meet, which is $r=0$.
    The most likely return (mode) is \textbf{0\%}.
\end{itemize}

\paragraph{Interpretation:}
We have found three different "central" values: Mode = 0\%, Median $\approx -0.261\%$, and Mean $\approx -0.333\%$.
The fact that Mean $<$ Median $<$ Mode indicates that the distribution is \textbf{left-skewed}. The "tail" of the distribution is longer on the left (negative returns/losses) side. This makes sense:
\begin{itemize}
    \item The most frequent outcome is a 0\% return (the peak).
    \item However, there is a 50\% chance of getting a return of -0.261\% or worse (the median).
    \item On average, over many days, you would expect to lose 0.333\% per day (the mean), as the larger possible losses pull the average down more than the smaller possible gains pull it up.
\end{itemize}

\subsection{Part (vi): Dollar Amount of VaR}
\paragraph{Reasoning:}
This is a direct application of the 5\% VaR percentage we found in part (iv) to the total investment amount.

\paragraph{Calculation:}
\begin{itemize}
    \item Investment Amount: \$250,000
    \item 5\% VaR (as a percentage loss): $L \approx 2.134\%$
    \item Dollar VaR = $L \times \text{Investment Amount}$
    \[ \$ \text{VaR} = 0.02134 \times \$250,000 = \$5,335 \]
\end{itemize}
\paragraph{Interpretation:}
With an investment of \$250,000 in this stock, there is a \textbf{5\% chance of losing \$5,335 or more} on any given day. This provides a concrete monetary value to the risk, which is far more tangible for an investor than a percentage.

\newpage

\section{In-depth Concepts}

Here are more detailed explanations of the key concepts used in this exercise, as covered in the script.

\hypertarget{note:pdf}{\subsection*{[1] Probability Density Function (PDF)}}
For a continuous random variable, the PDF, denoted $p(x)$ or $f(x)$, describes the relative likelihood of the variable taking on a particular value. Unlike a probability mass function (pmf) for discrete variables, the value of the PDF at a specific point $x$ is not the probability of that point occurring (which is always zero for continuous variables). Instead, the PDF's value represents a density. The probability of the variable falling within an interval $[a, b]$ is found by integrating the PDF over that interval: $P(a \le X \le b) = \int_a^b p(x) \,dx$. A valid PDF must be non-negative everywhere and its integral over the entire real line must equal 1 (\textit{Definition 1.39}).

\hypertarget{note:cdf}{\subsection*{[2] Cumulative Distribution Function (CDF)}}
The CDF, denoted $F(x)$, gives the probability that a random variable $X$ is less than or equal to a certain value $x$. It is defined as $F(x) = P(X \le x)$. For a continuous random variable, the CDF is the integral of the PDF from $-\infty$ to $x$. The CDF is a non-decreasing function that ranges from 0 to 1. It provides a complete description of the distribution. It is defined for both discrete and continuous random variables (\textit{Definition 1.21}).

\hypertarget{note:quantile}{\subsection*{[3] Quantile}}
The $q$-quantile of a distribution (where $q$ is a probability between 0 and 1) is the value $x_q$ such that a random variable $X$ from this distribution will be less than or equal to $x_q$ with probability $q$. In terms of the CDF, it is the value $x_q$ for which $F(x_q) = q$. If the CDF is strictly increasing and continuous, the quantile is unique and can be found using the inverse CDF: $x_q = F^{-1}(q)$. Quantiles are fundamental for understanding the spread and shape of a distribution (\textit{Definition 2.20}).

\hypertarget{note:var}{\subsection*{[4] Value at Risk (VaR)}}
VaR is a statistical measure used to quantify the level of financial risk within a firm or investment portfolio over a specific time frame. The VaR at a confidence level $\alpha$ is the maximum potential loss that is not expected to be exceeded with a probability of $1-\alpha$. The problem defines it as the loss $L$ where the probability of a worse loss is $\alpha$, i.e., $P(R \le -L) = \alpha$. It is essentially an application of quantiles to risk management, transforming a probability into a tangible (and often monetary) amount of potential loss.

\hypertarget{note:mean}{\subsection*{[5] Expected Value (Mean)}}
The expected value of a random variable, denoted $E[X]$, is the long-run average value of repetitions of the experiment it represents. It is a weighted average of all possible values, where the weights are their probabilities (or probability densities). For a continuous random variable, it's calculated as $E[X] = \int_{-\infty}^{\infty} x \cdot p(x) \,dx$. The expected value is a measure of the distribution's central tendency, often thought of as its "center of mass" (\textit{Definition 2.1}).

\hypertarget{note:median}{\subsection*{[6] Median}}
The median of a distribution is a specific quantile: the 0.5-quantile. It is the value that divides the distribution into two equal halves. There is a 50\% probability of the random variable being above the median and a 50\% probability of it being below. For a symmetric distribution, the mean and median are the same. For skewed distributions, the median is often considered a more robust measure of central tendency than the mean because it is not affected by extreme outliers.

\hypertarget{note:mode}{\subsection*{[7] Mode}}
The mode of a distribution is the value that appears most frequently. For a continuous distribution, the mode is the value $x$ where the PDF reaches its maximum value. It represents the peak of the distribution. A distribution can have one mode (unimodal), two modes (bimodal), or more (multimodal).

\end{document}
