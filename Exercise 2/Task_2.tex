\documentclass[11pt,a4paper]{article}

% --- PREAMBLE ---
% Preamble: Loading necessary packages
\usepackage[utf8]{inputenc}
\usepackage[a4paper, margin=1in]{geometry}
\usepackage{amsmath, amssymb, amsfonts}
\usepackage{graphicx}

% CORRECTED: hyperref package is generally best loaded last
\usepackage{hyperref}

% Hyperlink setup for clickable notes
\hypersetup{
    colorlinks=true,
    linkcolor=blue,
    urlcolor=blue,
    citecolor=red
}

% Author and Title Information
\title{Exercise Walkthrough: Commutativity of $\sigma$-Algebra Generation and Restriction}
\author{Justin Lanfermann}
\date{25. June 2025}

\begin{document}

\maketitle

\begin{abstract}
    This document provides a detailed, step-by-step walkthrough for an exercise from the Discrete Probability Theory course. The exercise demonstrates a key property of $\sigma$-algebras: that generating a $\sigma$-algebra from a restricted system of sets is equivalent to restricting the $\sigma$-algebra generated from the original system. We will leverage concepts from the lecture script, including measurable maps and inverse images, to construct a clear and formal proof.
\end{abstract}

\section{Overview and Goal}

The exercise asks us to prove the following statement:
\vspace{1em}

\noindent\textbf{Theorem.} Let $\Omega \neq \emptyset$ be a set, $A \subseteq \Omega$ with $A \neq \emptyset$, and let $\mathcal{E} \subseteq \mathcal{P}(\Omega)$ be a collection of subsets of $\Omega$. Then it holds that
\[
    \sigma(\mathcal{E}|_A) = \sigma(\mathcal{E})|_A .
\]
\vspace{1em}

In plain language, this theorem is about whether two different procedures result in the same final collection of sets.
\begin{enumerate}
    \item \textbf{The Left-Hand Side (LHS): $\sigma(\mathcal{E}|_A)$}. This means we first take our collection of "generator" sets $\mathcal{E}$ and restrict every set in it to $A$ (by intersecting each set with $A$). This gives us a new collection of sets, $\mathcal{E}|_A$, which are all subsets of $A$. Then, we find the smallest $\sigma$-algebra \emph{on the set A} that contains all these restricted sets.

    \item \textbf{The Right-Hand Side (RHS): $\sigma(\mathcal{E})|_A$}. This means we first take our original collection $\mathcal{E}$ and find the smallest $\sigma$-algebra \emph{on the set $\Omega$} that contains it. This gives us $\sigma(\mathcal{E})$. Then, we take this entire $\sigma$-algebra and restrict it to $A$ (again, by intersecting every set in it with $A$).
\end{enumerate}
The theorem states that these two procedures are equivalent. The order of "generating" and "restricting" doesn't matter. The provided solution sketch uses a clever and powerful tool: the \textbf{natural injection map} and its inverse. Let's build up the necessary concepts before diving into the proof.

\section{Preliminaries and Definitions}

To follow the proof, we need to be crystal clear on a few definitions, most of which are from the lecture script.

\paragraph{$\sigma$-Algebra (Definition 1.5).} A collection of subsets $\mathcal{A} \subseteq \mathcal{P}(\Omega)$ is a $\sigma$-algebra on $\Omega$ if it contains the empty set, is closed under complements, and is closed under countable unions.

\paragraph{Generated $\sigma$-Algebra (Lemma 1.7).} For any system of subsets $\mathcal{E} \subseteq \mathcal{P}(\Omega)$, the $\sigma$-algebra generated by $\mathcal{E}$, denoted $\sigma(\mathcal{E})$, is the smallest $\sigma$-algebra on $\Omega$ that contains $\mathcal{E}$. It is the intersection of all $\sigma$-algebras containing $\mathcal{E}$.

\paragraph{Restriction of a System of Sets.} The exercise uses the notation $\mathcal{E}|_A$. Let's define it formally. For a system of subsets $\mathcal{E} \subseteq \mathcal{P}(\Omega)$ and a set $A \subseteq \Omega$, the \textbf{restriction of $\mathcal{E}$ to $A$} is a collection of subsets of $A$ defined as:
\[
    \mathcal{E}|_A := \{E \cap A \mid E \in \mathcal{E}\}.
\]
This is analogous to the \textbf{induced $\sigma$-algebra} from Lemma 1.14 in the script, which uses the notation $\mathcal{A}|_B = \{A \cap B \mid A \in \mathcal{A}\}$ for an existing $\sigma$-algebra $\mathcal{A}$. The principle is the same: intersect every set in the collection with the subspace.

% CORRECTED: The unused hypertarget was here. I removed it.

\paragraph{The Natural Injection Map. \hyperref[exp3]{[1]}} The solution's key insight is to rephrase the "restriction" operation (set intersection) in terms of an inverse map. We define the \textbf{natural injection} (or inclusion) map from $A$ into $\Omega$ as:
\[
    \iota_A: A \to \Omega, \quad \text{where} \quad \iota_A(a) = a \quad \text{for all } a \in A.
\]
This map simply takes an element from the subset $A$ and views it as an element of the larger set $\Omega$. The magic happens when we look at its inverse. For any set $B \subseteq \Omega$, the inverse image of $B$ under $\iota_A$ is:
\[
    \iota_A^{-1}(B) = \{a \in A \mid \iota_A(a) \in B\} = \{a \in A \mid a \in B\} = A \cap B.
\]
This gives us a powerful bridge: restricting a set $B$ to $A$ is identical to taking the inverse image of $B$ under the natural injection map $\iota_A$.

\section{The Step-by-Step Proof}

Now we have all the tools to follow the solution. We want to show that $\sigma(\mathcal{E}|_A) = \sigma(\mathcal{E})|_A$. We will start with the LHS and transform it step-by-step into the RHS.

\begin{align}
    \sigma(\mathcal{E}|_A) &= \sigma\left( \{E \cap A \mid E \in \mathcal{E}\} \right) \label{step1} \\
    &= \sigma\left( \{\iota_A^{-1}(E) \mid E \in \mathcal{E}\} \right) \label{step2} \\
    &= \sigma\left( \iota_A^{-1}(\mathcal{E}) \right) \label{step3} \\
    &= \iota_A^{-1}\left( \sigma(\mathcal{E}) \right) \label{step4} \\
    &= \{A \cap B \mid B \in \sigma(\mathcal{E})\} \label{step5} \\
    &= \sigma(\mathcal{E})|_A \label{step6}
\end{align}

Let's break down the reasoning for each step.

\paragraph{Step \eqref{step1}: Expanding the definition.}
Here, we simply apply the definition of the restriction of a system of sets, $\mathcal{E}|_A$, which we established in the preliminaries. This step just makes the notation explicit.

\paragraph{Step \eqref{step2}: Introducing the injection map.}
This step uses the crucial property of the natural injection map $\iota_A$. As we showed, for any set $E \subseteq \Omega$, its intersection with $A$ is exactly its inverse image under $\iota_A$. So, we replace every instance of $E \cap A$ with $\iota_A^{-1}(E)$. We are just rephrasing the problem.

\paragraph{Step \eqref{step3}: Simplifying notation.}
This is a purely notational change. The set $\{\iota_A^{-1}(E) \mid E \in \mathcal{E}\}$ is more compactly written as $\iota_A^{-1}(\mathcal{E})$, which means "the collection of all inverse images of sets in $\mathcal{E}$".

\paragraph{Step \eqref{step4}: The core argument. \hyperref[exp2]{[2]}}
This is the most important step of the proof. It states that generating a $\sigma$-algebra from a collection of inverse images is the same as taking the inverse image of the $\sigma$-algebra generated from the original collection. Formally, for a map $f: X \to Y$ and a generator system $\mathcal{E} \subseteq \mathcal{P}(Y)$, it holds that $\sigma(f^{-1}(\mathcal{E})) = f^{-1}(\sigma(\mathcal{E}))$.
The exercise hint says to "use the result from Exercise 1", which is precisely this property. This result is a standard theorem in measure theory because the inverse image operation preserves all the necessary set operations (unions, intersections, complements) required to build a $\sigma$-algebra.

\paragraph{Step \eqref{step5}: Returning to intersections.}
Here, we reverse what we did in Step \eqref{step2}. We expand the compact notation $\iota_A^{-1}(\sigma(\mathcal{E}))$ back into its definition: it is the collection of all sets $A \cap B$ where $B$ is a set from the $\sigma$-algebra $\sigma(\mathcal{E})$.

\paragraph{Step \eqref{step6}: Applying the definition of restriction.}
Finally, we recognize that the expression $\{A \cap B \mid B \in \sigma(\mathcal{E})\}$ is exactly the definition of the restriction of the $\sigma$-algebra $\sigma(\mathcal{E})$ to the subset $A$. This is the RHS we wanted to reach.

This completes the proof. We have shown that $\sigma(\mathcal{E}|_A) = \sigma(\mathcal{E})|_A$.

\section{Check Your Understanding}

To solidify this concept, try to work through a concrete example.

\textbf{Exercise:} Let $\Omega = \{1, 2, 3, 4, 5, 6\}$ (a standard die roll). Let $A = \{1, 3, 5\}$ be the event of rolling an odd number. Let the generator system be $\mathcal{E} = \{\{1, 2\}, \{4, 5\}\}$.
\begin{enumerate}
    \item Explicitly compute the left-hand side:
    \begin{itemize}
        \item First find $\mathcal{E}|_A$.
        \item Then find $\sigma(\mathcal{E}|_A)$, which will be a $\sigma$-algebra on $A$.
    \end{itemize}
    \item Explicitly compute the right-hand side:
    \begin{itemize}
        \item First find $\sigma(\mathcal{E})$, which will be a $\sigma$-algebra on $\Omega$.
        \item Then find $\sigma(\mathcal{E})|_A$.
    \end{itemize}
    \item Verify that your results from both parts are identical.
\end{enumerate}

\section{Summary and Takeaways}

\begin{itemize}
    \item \textbf{Main Result:} The operations of generating a $\sigma$-algebra and restricting to a subspace are commutative. This is a useful and elegant property.
    \item \textbf{Proof Technique:} The key to the formal proof was translating the geometric idea of "restriction" or "intersection" into the language of functions and their inverse images using the natural injection map $\iota_A$.
    \item \textbf{The Power of Inverse Images:} The property $\sigma(f^{-1}(\mathcal{E})) = f^{-1}(\sigma(\mathcal{E}))$ is fundamental for working with measurable functions and induced structures. It essentially states that the inverse image is "well-behaved" with respect to the operations that define a $\sigma$-algebra.
\end{itemize}

This result is a building block for more advanced topics, like understanding the distributions of random variables, which are themselves defined as measurable maps.

\newpage
\section*{In-Depth Explanations}

% CORRECTED: Using \phantomsection to create an anchor for the unnumbered subsection
\phantomsection
\subsection*{[1] The Natural Injection Map}\label{exp3}
The natural injection map is a simple but powerful concept. When we have a subset $A \subseteq \Omega$, the map $\iota_A: A \to \Omega$ defined by $\iota_A(a) = a$ formalizes the idea that $A$ is "sitting inside" $\Omega$.

The real utility comes from its inverse image, $\iota_A^{-1}$. For any subset $B \subseteq \Omega$, the inverse image $\iota_A^{-1}(B)$ asks: "Which elements \emph{in the domain A} are mapped into the set $B$?"
Since the mapping rule is just identity ($\iota_A(a) = a$), an element $a \in A$ is mapped into $B$ if and only if $a$ is itself an element of $B$. Therefore, the elements we are looking for are those that are in $A$ AND in $B$. This is precisely the definition of the set intersection $A \cap B$.

This turns a set operation ($A \cap \cdot$) into a map operation ($\iota_A^{-1}(\cdot)$), allowing us to use powerful theorems about maps.

\phantomsection
\subsection*{[2] Commutativity of Inverse Image and $\sigma$-Algebra Generation}\label{exp2}
The property $\sigma(f^{-1}(\mathcal{E})) = f^{-1}(\sigma(\mathcal{E}))$ is the cornerstone of the proof. Let's briefly sketch why it's true for a general map $f: X \to Y$ and a generator $\mathcal{E} \subseteq \mathcal{P}(Y)$.

The proof is a classic "show two sets are equal by showing each is a subset of the other" argument.

\begin{enumerate}
    \item \textbf{Show $\sigma(f^{-1}(\mathcal{E})) \subseteq f^{-1}(\sigma(\mathcal{E}))$}:
    We know that $\mathcal{E} \subseteq \sigma(\mathcal{E})$. Applying the inverse map to both sides gives $f^{-1}(\mathcal{E}) \subseteq f^{-1}(\sigma(\mathcal{E}))$.
    Now, we can show that $f^{-1}(\sigma(\mathcal{E}))$ is itself a $\sigma$-algebra on $X$. This is because the inverse image operation commutes with set operations:
    \begin{itemize}
        \item $f^{-1}(Y) = X$ (and $f^{-1}(\emptyset) = \emptyset$).
        \item $f^{-1}(B^c) = (f^{-1}(B))^c$.
        \item $f^{-1}(\bigcup_i B_i) = \bigcup_i f^{-1}(B_i)$.
    \end{itemize}
    Since $f^{-1}(\sigma(\mathcal{E}))$ is a $\sigma$-algebra that contains the generator system $f^{-1}(\mathcal{E})$, it must be larger than or equal to the \emph{smallest} $\sigma$-algebra containing $f^{-1}(\mathcal{E})$, which is $\sigma(f^{-1}(\mathcal{E}))$. Thus, the inclusion holds.

    \item \textbf{Show $f^{-1}(\sigma(\mathcal{E})) \subseteq \sigma(f^{-1}(\mathcal{E}))$}:
    This direction is more involved. It requires showing that the collection of sets $\{B \subseteq Y \mid f^{-1}(B) \in \sigma(f^{-1}(\mathcal{E}))\}$ is a $\sigma$-algebra on $Y$ that contains $\mathcal{E}$. By the minimality of $\sigma(\mathcal{E})$, this implies $\sigma(\mathcal{E})$ is a subset of this collection, which proves the inclusion.
\end{enumerate}
For our purposes, accepting this as a given theorem (as instructed by the exercise) is sufficient.

\end{document}
