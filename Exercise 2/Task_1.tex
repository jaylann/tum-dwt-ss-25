\documentclass[11pt,a4paper]{article}

% --- PACKAGES ---
\usepackage[a4paper, margin=2.5cm]{geometry}
\usepackage{amsmath, amssymb, amsthm}
\usepackage[utf8]{inputenc}
\usepackage{hyperref}
\usepackage[dvipsnames]{xcolor}
\usepackage{bookmark}

% --- HYPERREF SETUP ---
\hypersetup{
    colorlinks=true,
    linkcolor=MidnightBlue,
    urlcolor=RoyalBlue,
    citecolor=ForestGreen,
    linktoc=all
}

% --- THEOREM-LIKE ENVIRONMENTS ---
\newtheoremstyle{exercise_style}
  {\topsep}   % Space above
  {\topsep}   % Space below
  {\itshape}  % Body font
  {}          % Indent amount
  {\bfseries} % Theorem head font
  {.}         % Punctuation after theorem head
  {.5em}      % Space after theorem head
  {}          % Theorem head spec (can be left empty, meaning `normal')
\theoremstyle{exercise_style}
\newtheorem*{exercise}{Exercise}

\theoremstyle{definition}
\newtheorem*{solution}{Solution}

% --- METADATA ---
\title{\bfseries Exercise Walkthrough: Measurability and Generators}
\author{Justin Lanfermann}
\date{25. June of 2025}

% --- DOCUMENT ---
\begin{document}

\maketitle

\begin{abstract}
    This document provides a detailed, step-by-step walkthrough of a fundamental exercise in probability theory. The exercise establishes that to verify the measurability of a function, it is sufficient to check the property on a generating set of the codomain's $\sigma$-algebra. We will break down the proof into its core components, explain the reasoning behind each step, and connect the logic to key definitions from the "Discrete Probability Theory" script.
\end{abstract}

\tableofcontents
\newpage

\section{Overview and Motivation}

Hello! Today, we're going to tackle an exercise that might seem a bit abstract at first, but its result is a cornerstone for working with measurable functions in practice.

In probability theory, we need functions to be \textbf{measurable} \hyperlink{concept:measurable_function}{[3]} to ensure we can meaningfully talk about the probabilities of events involving them. For instance, if $X$ is a random variable, we want to be able to compute $P(X \in A)$. The definition of measurability (Def. 1.45 in the script) requires us to check a condition for \textit{every} set in the target $\sigma$-algebra. This can be a daunting task, as a $\sigma$-algebra can contain an enormous number of very complex sets.

This exercise provides us with a powerful shortcut. It shows that we only need to check the measurability condition for a smaller, simpler collection of sets, the so-called \textbf{generator} \hyperlink{concept:generator}{[2]} of the $\sigma$-algebra. This is analogous to linear algebra, where to check if a linear map is zero, you only need to check its action on the basis vectors, not on every vector in the space.

Let's state the problem and then dive into the solution.

\section{The Exercise}

\begin{exercise}
Let $(X, \mathcal{A}_X)$ and $(Y, \mathcal{A}_Y)$ be measurable spaces, $f : X \to Y$ a map, and $\mathcal{E}_Y \subseteq \mathcal{A}_Y$ arbitrary. Show that
\begin{equation} \label{eq:1}
    \sigma(f^{-1}(\mathcal{E}_Y)) = f^{-1}(\sigma(\mathcal{E}_Y)).
\end{equation}
Further, show that
\begin{equation} \label{eq:2}
    f \text{ is } \mathcal{A}_X\text{-}\sigma(\mathcal{E}_Y)\text{-measurable} \iff f^{-1}(E) \in \mathcal{A}_X \text{ for all } E \in \mathcal{E}_Y.
\end{equation}
With that, finally show that for $\sigma(\mathcal{E}_Y) = \mathcal{A}_Y$ it holds that
\begin{equation} \label{eq:3}
    f \text{ is } \mathcal{A}_X\text{-}\mathcal{A}_Y\text{-measurable} \iff f^{-1}(\mathcal{E}_Y) \subseteq \mathcal{A}_X.
\end{equation}
\end{exercise}

\begin{solution}
We will prove each part sequentially. The first part is a technical result about sets and functions that we'll then use to prove the main statement about measurability in the second and third parts.
\end{solution}

\subsection{Part 1: The Generator-Preimage Identity}
Our goal is to prove the set equality $\sigma(f^{-1}(\mathcal{E}_Y)) = f^{-1}(\sigma(\mathcal{E}_Y))$. To prove that two sets are equal, we must show that each is a subset of the other.

\subsubsection{Proof of "$\subseteq$": $\sigma(f^{-1}(\mathcal{E}_Y)) \subseteq f^{-1}(\sigma(\mathcal{E}_Y))$}
This is the more straightforward direction. Our strategy is to show that the set on the right, $f^{-1}(\sigma(\mathcal{E}_Y))$, is a $\sigma$-algebra that contains the set $f^{-1}(\mathcal{E}_Y)$.

\begin{enumerate}
    \item \textbf{The Right-Hand Side is a $\sigma$-algebra:} First, let's establish that for any $\sigma$-algebra $\mathcal{A}$ on $Y$, its preimage $f^{-1}(\mathcal{A}) = \{f^{-1}(A) | A \in \mathcal{A}\}$ is a $\sigma$-algebra on $X$. This is a standard result that relies on the fact that the preimage operation commutes with set operations (complement, union, intersection) \hyperlink{concept:preimage}{[4]}. Since $\sigma(\mathcal{E}_Y)$ is by definition a \textbf{$\sigma$-algebra} \hyperlink{concept:sigma_algebra}{[1]} on $Y$, its preimage $f^{-1}(\sigma(\mathcal{E}_Y))$ is a $\sigma$-algebra on $X$.

    \item \textbf{It Contains the Generating Set:} By definition of a generator (Def. 1.7 in the script), we know that $\mathcal{E}_Y \subseteq \sigma(\mathcal{E}_Y)$. Applying the preimage function $f^{-1}$ to both sides of this inclusion gives us:
    \[
        f^{-1}(\mathcal{E}_Y) \subseteq f^{-1}(\sigma(\mathcal{E}_Y)).
    \]
    This holds because the preimage operation preserves subset relations.

    \item \textbf{Conclusion:} We have a set $f^{-1}(\mathcal{E}_Y)$. The term $\sigma(f^{-1}(\mathcal{E}_Y))$ is, by definition, the \textit{smallest} $\sigma$-algebra on $X$ that contains $f^{-1}(\mathcal{E}_Y)$.
    From steps 1 and 2, we know that $f^{-1}(\sigma(\mathcal{E}_Y))$ is \textit{a} $\sigma$-algebra on $X$ that contains $f^{-1}(\mathcal{E}_Y)$. Therefore, the smallest such $\sigma$-algebra must be a subset of this one.
    \[
        \sigma(f^{-1}(\mathcal{E}_Y)) \subseteq f^{-1}(\sigma(\mathcal{E}_Y)).
    \]
\end{enumerate}

\subsubsection{Proof of "$\supseteq$": $f^{-1}(\sigma(\mathcal{E}_Y)) \subseteq \sigma(f^{-1}(\mathcal{E}_Y))$}
This direction is more involved. The strategy here is to use a common proof technique for generated $\sigma$-algebras: define a "helper" collection of sets, show it's a $\sigma$-algebra containing the generator, and then draw conclusions.

\begin{enumerate}
    \item \textbf{Define a Helper Collection:} Let's define a collection of sets in $Y$:
    \[
        \tilde{\mathcal{A}} = \{ A' \in \sigma(\mathcal{E}_Y) \mid f^{-1}(A') \in \sigma(f^{-1}(\mathcal{E}_Y)) \}.
    \]
    In plain English, $\tilde{\mathcal{A}}$ consists of all the sets from the "big" $\sigma$-algebra $\sigma(\mathcal{E}_Y)$ whose preimages land in the "small" $\sigma$-algebra $\sigma(f^{-1}(\mathcal{E}_Y))$. Our goal is to show that this collection $\tilde{\mathcal{A}}$ is actually equal to $\sigma(\mathcal{E}_Y)$.

    \item \textbf{Show $\tilde{\mathcal{A}}$ is a $\sigma$-algebra:} We check the three conditions from Def. 1.5 in the script.
        \begin{itemize}
            \item \textbf{Contains Y?} We need to check if $Y \in \tilde{\mathcal{A}}$. The condition is $f^{-1}(Y) \in \sigma(f^{-1}(\mathcal{E}_Y))$. Since $f^{-1}(Y) = X$, and any $\sigma$-algebra on $X$ must contain $X$, this is true. So $Y \in \tilde{\mathcal{A}}$.
            \item \textbf{Closed under complements?} Let $A' \in \tilde{\mathcal{A}}$. This means $f^{-1}(A') \in \sigma(f^{-1}(\mathcal{E}_Y))$. We need to check if $(A')^c \in \tilde{\mathcal{A}}$. The condition is $f^{-1}((A')^c) \in \sigma(f^{-1}(\mathcal{E}_Y))$. Since $f^{-1}((A')^c) = (f^{-1}(A'))^c$ \hyperlink{concept:preimage}{[4]} and $\sigma(f^{-1}(\mathcal{E}_Y))$ is a $\sigma$-algebra (closed under complements), this is true.
            \item \textbf{Closed under countable unions?} Let $A'_1, A'_2, \dots \in \tilde{\mathcal{A}}$. This means $f^{-1}(A'_i) \in \sigma(f^{-1}(\mathcal{E}_Y))$ for all $i$. We need to check if $\bigcup_i A'_i \in \tilde{\mathcal{A}}$. The condition is $f^{-1}(\bigcup_i A'_i) \in \sigma(f^{-1}(\mathcal{E}_Y))$. Since $f^{-1}(\bigcup_i A'_i) = \bigcup_i f^{-1}(A'_i)$ \hyperlink{concept:preimage}{[4]} and $\sigma(f^{-1}(\mathcal{E}_Y))$ is a $\sigma$-algebra (closed under countable unions), this is also true.
        \end{itemize}
    Thus, $\tilde{\mathcal{A}}$ is a $\sigma$-algebra on $Y$.

    \item \textbf{Show $\mathcal{E}_Y \subseteq \tilde{\mathcal{A}}$:} Let's take an arbitrary set $E \in \mathcal{E}_Y$. To show that $E \in \tilde{\mathcal{A}}$, we must check if it satisfies the condition in the definition of $\tilde{\mathcal{A}}$, which is: $f^{-1}(E) \in \sigma(f^{-1}(\mathcal{E}_Y))$. This is true by the very definition of a generated $\sigma$-algebra, because $f^{-1}(E)$ is an element of the generating set $f^{-1}(\mathcal{E}_Y)$, which is a subset of $\sigma(f^{-1}(\mathcal{E}_Y))$. So, $\mathcal{E}_Y \subseteq \tilde{\mathcal{A}}$.

    \item \textbf{Conclusion:} From steps 2 and 3, we have shown that $\tilde{\mathcal{A}}$ is a $\sigma$-algebra on $Y$ that contains the generating set $\mathcal{E}_Y$. Since $\sigma(\mathcal{E}_Y)$ is the \textit{smallest} such $\sigma$-algebra, it must be that $\sigma(\mathcal{E}_Y) \subseteq \tilde{\mathcal{A}}$.

    But by its definition, $\tilde{\mathcal{A}}$ is already a subset of $\sigma(\mathcal{E}_Y)$. The only way both $\sigma(\mathcal{E}_Y) \subseteq \tilde{\mathcal{A}}$ and $\tilde{\mathcal{A}} \subseteq \sigma(\mathcal{E}_Y)$ can be true is if they are equal: $\sigma(\mathcal{E}_Y) = \tilde{\mathcal{A}}$.

    Finally, what does this mean? It means for any set $A' \in \sigma(\mathcal{E}_Y)$, it is also in $\tilde{\mathcal{A}}$, which by definition implies $f^{-1}(A') \in \sigma(f^{-1}(\mathcal{E}_Y))$. This is precisely the inclusion we wanted to show:
    \[
        f^{-1}(\sigma(\mathcal{E}_Y)) \subseteq \sigma(f^{-1}(\mathcal{E}_Y)).
    \]
\end{enumerate}
Since we have shown both inclusions, we have proven that $\sigma(f^{-1}(\mathcal{E}_Y)) = f^{-1}(\sigma(\mathcal{E}_Y))$.
\hfill $\qed$

\subsection{Part 2: The Measurability Equivalence}

Now we use our result from Part 1 to prove the main statement about measurability. We need to prove an "if and only if" statement.

\subsubsection{Proof of "$\Rightarrow$"}
This direction is straightforward.
\begin{enumerate}
    \item \textbf{Assumption:} We assume $f$ is $\mathcal{A}_X$-$\sigma(\mathcal{E}_Y)$-measurable.
    \item \textbf{Definition of Measurability (Def. 1.45):} This means that for every set $A' \in \sigma(\mathcal{E}_Y)$, its preimage $f^{-1}(A')$ must be in $\mathcal{A}_X$.
    \item \textbf{Goal:} We want to show that $f^{-1}(E) \in \mathcal{A}_X$ for all $E \in \mathcal{E}_Y$.
    \item \textbf{Argument:} By definition, the generator $\mathcal{E}_Y$ is a subset of the generated $\sigma$-algebra $\sigma(\mathcal{E}_Y)$. So, any set $E \in \mathcal{E}_Y$ is also a set in $\sigma(\mathcal{E}_Y)$. Since our assumption holds for \textit{all} sets in $\sigma(\mathcal{E}_Y)$, it must also hold for all sets in $\mathcal{E}_Y$. The implication is direct.
\end{enumerate}

\subsubsection{Proof of "$\Leftarrow$"}
This is the more useful direction, where we get the "shortcut".
\begin{enumerate}
    \item \textbf{Assumption:} We assume that $f^{-1}(E) \in \mathcal{A}_X$ for all $E \in \mathcal{E}_Y$. This can be written more concisely as the set inclusion $f^{-1}(\mathcal{E}_Y) \subseteq \mathcal{A}_X$.
    \item \textbf{Goal:} We want to show that $f$ is $\mathcal{A}_X$-$\sigma(\mathcal{E}_Y)$-measurable. This means we must show that $f^{-1}(A') \in \mathcal{A}_X$ for all $A' \in \sigma(\mathcal{E}_Y)$, or concisely, $f^{-1}(\sigma(\mathcal{E}_Y)) \subseteq \mathcal{A}_X$.
    \item \textbf{Argument:}
        \begin{itemize}
            \item From our assumption $f^{-1}(\mathcal{E}_Y) \subseteq \mathcal{A}_X$, and since $\mathcal{A}_X$ is itself a $\sigma$-algebra, it must contain the smallest $\sigma$-algebra generated by $f^{-1}(\mathcal{E}_Y)$. That is:
            \[
                \sigma(f^{-1}(\mathcal{E}_Y)) \subseteq \sigma(\mathcal{A}_X) = \mathcal{A}_X.
            \]
            \item Now, we use our hard-won identity from Part 1: $\sigma(f^{-1}(\mathcal{E}_Y)) = f^{-1}(\sigma(\mathcal{E}_Y))$.
            \item Substituting this into the previous line gives us:
            \[
                f^{-1}(\sigma(\mathcal{E}_Y)) \subseteq \mathcal{A}_X.
            \]
            \item This is precisely the definition of $f$ being $\mathcal{A}_X$-$\sigma(\mathcal{E}_Y)$-measurable.
        \end{itemize}
\end{enumerate}
This completes the proof of the equivalence. \hfill $\qed$

\subsection{Part 3: The Corollary}
This final part is a direct application of what we just proved.
\begin{enumerate}
    \item \textbf{Setup:} We are given that $\sigma(\mathcal{E}_Y) = \mathcal{A}_Y$. This means $\mathcal{E}_Y$ is a generator for the entire target $\sigma$-algebra $\mathcal{A}_Y$.
    \item \textbf{Goal:} We want to show that $f$ is $\mathcal{A}_X$-$\mathcal{A}_Y$-measurable if and only if $f^{-1}(\mathcal{E}_Y) \subseteq \mathcal{A}_X$.
    \item \textbf{Argument:} Let's take the equivalence we proved in Part 2:
    \[
        f \text{ is } \mathcal{A}_X\text{-}\sigma(\mathcal{E}_Y)\text{-measurable} \iff f^{-1}(E) \in \mathcal{A}_X \text{ for all } E \in \mathcal{E}_Y.
    \]
    Now, we simply substitute our setup condition $\sigma(\mathcal{E}_Y) = \mathcal{A}_Y$ into the left-hand side. The statement becomes:
    \[
        f \text{ is } \mathcal{A}_X\text{-}\mathcal{A}_Y\text{-measurable} \iff f^{-1}(E) \in \mathcal{A}_X \text{ for all } E \in \mathcal{E}_Y.
    \]
    The right-hand side, "$f^{-1}(E) \in \mathcal{A}_X$ for all $E \in \mathcal{E}_Y$", is just the definition of the set inclusion $f^{-1}(\mathcal{E}_Y) \subseteq \mathcal{A}_X$.
    So, the equivalence holds.
\end{enumerate}
\hfill $\qed$

\section{Summary and Takeaways}

We have successfully proven a very important result in measure theory. The main takeaway is this:

\begin{center}
\parbox{0.8\textwidth}{%
\textbf{Key Result:} To prove that a function $f: (X, \mathcal{A}_X) \to (Y, \mathcal{A}_Y)$ is measurable, you do not need to check that the preimage of \textit{every} set in $\mathcal{A}_Y$ belongs to $\mathcal{A}_X$. Instead, you only need to find a \textit{generating set} $\mathcal{E}_Y$ for $\mathcal{A}_Y$ (i.e., $\sigma(\mathcal{E}_Y) = \mathcal{A}_Y$) and check the condition just for the sets in $\mathcal{E}_Y$.
}
\end{center}

\paragraph{Why is this useful?} Consider a continuous function $f: \mathbb{R} \to \mathbb{R}$. We want to show it is Borel-measurable, i.e., measurable with respect to the Borel $\sigma$-algebra $\mathcal{B}$ on both the domain and codomain. The Borel $\sigma$-algebra is generated by the collection of all open intervals $(a, b)$. Checking measurability for every bizarre Borel set would be impossible. But thanks to our result, we only need to check that for any open interval $(a, b)$, the preimage $f^{-1}((a, b))$ is a Borel set. Since $f$ is continuous, we know from analysis that the preimage of an open set is always an open set. And since all open sets are, by definition, in the Borel $\sigma$-algebra, the condition is satisfied. The proof becomes trivial!

\paragraph{Quick Question for You:}
Based on this, can you prove that if $X$ is a real-valued random variable (i.e., a measurable function from $(\Omega, \mathcal{A})$ to $(\mathbb{R}, \mathcal{B})$), then the function $Y = X^2$ is also a random variable? What would you choose as the generating set for $\mathcal{B}$ on the codomain to make the proof easiest?

\newpage

\section{In-depth Concepts}
\label{sec:concepts}
Here are more detailed explanations of the key concepts used in this walkthrough.

\begin{description}
    \item[\hypertarget{concept:sigma_algebra}{[1] What is a $\sigma$-algebra?}] A $\sigma$-algebra (or sigma-field) $\mathcal{A}$ on a set $X$ is a collection of subsets of $X$ that is "stable" enough for us to build a consistent theory of measure (like length, volume, or probability). It must satisfy three properties (Def. 1.5):
    \begin{enumerate}
        \item It contains the whole space: $X \in \mathcal{A}$.
        \item It is closed under complements: If $A \in \mathcal{A}$, then its complement $A^c = X \setminus A$ must also be in $\mathcal{A}$.
        \item It is closed under countable unions: If $A_1, A_2, A_3, \dots$ is a sequence of sets in $\mathcal{A}$, then their union $\bigcup_{i=1}^{\infty} A_i$ must also be in $\mathcal{A}$.
    \end{enumerate}
    The sets inside $\mathcal{A}$ are called \textit{measurable sets}. In probability, this is our \textit{event space}.

    \item[\hypertarget{concept:generator}{[2] What is a generator of a $\sigma$-algebra?}] Often, a $\sigma$-algebra is too large and complex to describe by listing all its elements. A generator is a smaller, simpler collection of subsets $\mathcal{E}$ whose sets we consider fundamentally important. The $\sigma$-algebra generated by $\mathcal{E}$, denoted $\sigma(\mathcal{E})$, is defined as the \textit{smallest} $\sigma$-algebra that contains all the sets in $\mathcal{E}$ (Def. 1.7). For example, the Borel $\sigma$-algebra on $\mathbb{R}$, denoted $\mathcal{B}$, is generated by the collection of all open intervals.

    \item[\hypertarget{concept:measurable_function}{[3] What is a measurable function?}] A function $f: (X, \mathcal{A}_X) \to (Y, \mathcal{A}_Y)$ between two measurable spaces is called \textit{measurable} if it respects their structure (Def. 1.45). This means that for any measurable set $A_Y$ in the codomain $Y$ (i.e., $A_Y \in \mathcal{A}_Y$), its preimage $f^{-1}(A_Y)$ must be a measurable set in the domain $X$ (i.e., $f^{-1}(A_Y) \in \mathcal{A}_X$). In probability theory, a \textit{random variable} is simply a measurable function from the sample space $(\Omega, \mathcal{A})$ to a space of values, like $(\mathbb{R}, \mathcal{B})$. This property guarantees that questions like "what is the probability that the random variable $X$ has a value less than 5?" are well-posed, because $\{ \omega \in \Omega \mid X(\omega) < 5 \} = X^{-1}((-\infty, 5))$ is guaranteed to be an event in $\mathcal{A}$.

    \item[\hypertarget{concept:preimage}{[4] What are preimages and why are they so useful?}] The preimage (or inverse image) of a set $A_Y \subseteq Y$ under a function $f: X \to Y$ is the set of all points in the domain $X$ that map into $A_Y$. Formally, $f^{-1}(A_Y) = \{x \in X \mid f(x) \in A_Y\}$. A key reason preimages are central to measure theory is that they behave very well with respect to set operations:
    \begin{itemize}
        \item $f^{-1}(A_Y^c) = (f^{-1}(A_Y))^c$
        \item $f^{-1}(\bigcup_i A_{Y,i}) = \bigcup_i f^{-1}(A_{Y,i})$
        \item $f^{-1}(\bigcap_i A_{Y,i}) = \bigcap_i f^{-1}(A_{Y,i})$
    \end{itemize}
    This "commuting" property is exactly what allows us to prove that the preimage of a $\sigma$-algebra is also a $\sigma$-algebra, a crucial step in our proof above.
\end{description}

\end{document}
