\documentclass[11pt,a4paper]{article}

% --- PACKAGES ---
\usepackage[a4paper, margin=2.5cm]{geometry} % Set page margins
\usepackage{amsmath, amssymb, amsthm}       % For math symbols, environments
\usepackage{graphicx}                       % To include images
\usepackage[utf8]{inputenc}                 % Input encoding
\usepackage{hyperref}                       % For clickable links
\usepackage{xcolor}                         % For colors

% --- HYPERREF SETUP ---
\hypersetup{
    colorlinks=true,
    linkcolor=blue,
    filecolor=magenta,
    urlcolor=cyan,
    pdftitle={Exercise Walkthrough},
    pdfpagemode=FullScreen,
}

% --- THEOREM-LIKE ENVIRONMENTS ---
\newtheorem{theorem}{Theorem}
\newtheorem{exercise}{Exercise}
\newtheorem{definition}{Definition}
\newtheorem{solution}{Solution}

% --- DOCUMENT METADATA ---
\title{Exercise Walkthrough: \(\sigma\)-algebras from Mappings}
\author{Justin Lanfermann}
\date{25. June 2025}

% --- START OF DOCUMENT ---
\begin{document}

\maketitle

\begin{abstract}
    This document provides a step-by-step walkthrough for an exercise from the Discrete Probability Theory course. The exercise demonstrates that the pushforward and the inverse image of a \(\sigma\)-algebra under an arbitrary mapping are themselves \(\sigma\)-algebras. We will break down the proof into its constituent parts, explaining the reasoning behind each step and referencing key definitions from the course script.
\end{abstract}

\section{The Exercise}

\begin{exercise}
Let \(\Omega_1, \Omega_2\) be nonempty sets, \(f : \Omega_1 \to \Omega_2\) an arbitrary mapping, and let \(\mathcal{A}\) and \(\mathcal{B}\) be \(\sigma\)-algebras over \(\Omega_1\) and \(\Omega_2\) respectively. Then the inverse image of \(\mathcal{B}\) under \(f\) and the image or push-forward of \(\mathcal{A}\) under \(f\), defined as
\[ f^{-1}(\mathcal{B}) := \{f^{-1}(B) \mid B \in \mathcal{B}\} \quad \text{and} \quad f_*(\mathcal{A}) := \{B \subseteq \Omega_2 \mid f^{-1}(B) \in \mathcal{A}\} \]
are \(\sigma\)-algebras over \(\Omega_1\) and \(\Omega_2\) respectively.
\end{exercise}

\hrule
\vspace{1em}

\subsection*{Overview of the Proof Strategy}
Our goal is to prove that two different sets of sets, \(f_*(\mathcal{A})\) and \(f^{-1}(\mathcal{B})\), are \(\sigma\)-algebras. To do this for each one, we just need to check if it satisfies the three conditions laid out in the definition of a \(\sigma\)-algebra \hyperlink{def:sigma-algebra}{[1]}.

A set system \(\mathcal{F}\) on a space \(\Omega\) is a \(\sigma\)-algebra if:
\begin{enumerate}
    \item[\textbf{(i)}] It contains the empty set: \(\emptyset \in \mathcal{F}\).
    \item[\textbf{(ii)}] It is closed under complementation: If \(A \in \mathcal{F}\), then its complement \(A^c = \Omega \setminus A\) is also in \(\mathcal{F}\).
    \item[\textbf{(iii)}] It is closed under countable unions: If \(A_1, A_2, \dots\) is a sequence of sets in \(\mathcal{F}\), then their union \(\bigcup_{j=1}^{\infty} A_j\) is also in \(\mathcal{F}\).
\end{enumerate}
The core idea of the proof is to leverage the fact that we already know \(\mathcal{A}\) and \(\mathcal{B}\) are \(\sigma\)-algebras. We will translate the conditions for our new set systems back to conditions on \(\mathcal{A}\) or \(\mathcal{B}\) using the properties of the preimage \hyperlink{def:preimage}{[2]} of a function.

\newpage
\section{Part 1: Proving \(f_*(\mathcal{A})\) is a \(\sigma\)-algebra on \(\Omega_2\)}

Let's start with the pushforward algebra, \(f_*(\mathcal{A})\).
First, let's remind ourselves of its definition:
\[ f_*(\mathcal{A}) = \{B \subseteq \Omega_2 \mid f^{-1}(B) \in \mathcal{A}\} \]
In plain language, a set \(B\) from \(\Omega_2\) gets to be in our new collection \(f_*(\mathcal{A})\) if and only if its preimage, \(f^{-1}(B)\), is a member of the original \(\sigma\)-algebra \(\mathcal{A}\) on \(\Omega_1\). Our task is to show that this collection \(f_*(\mathcal{A})\) satisfies the three sigma-algebra axioms.

\vspace{1em}
\hrule
\vspace{1em}

\subsubsection*{Step 1.1: Checking for the Empty Set (Axiom i)}
\textbf{Goal:} Show that \(\emptyset \in f_*(\mathcal{A})\).

\textbf{Reasoning:} According to the definition of \(f_*(\mathcal{A})\), we need to check if the preimage of \(\emptyset\), which is \(f^{-1}(\emptyset)\), is in \(\mathcal{A}\).

\textbf{Proof Step:}
The preimage of the empty set is the set of all elements in \(\Omega_1\) that map to the empty set. No element can do this, so the preimage is always the empty set:
\[ f^{-1}(\emptyset) = \emptyset \]
We know that \(\mathcal{A}\) is a \(\sigma\)-algebra on \(\Omega_1\). By definition (Axiom i), every \(\sigma\)-algebra must contain the empty set. Therefore, \(f^{-1}(\emptyset) = \emptyset \in \mathcal{A}\).
Since the preimage of \(\emptyset\) is in \(\mathcal{A}\), by the definition of \(f_*(\mathcal{A})\), we conclude that \(\emptyset \in f_*(\mathcal{A})\).

\vspace{1em}
\hrule
\vspace{1em}

\subsubsection*{Step 1.2: Checking Closure under Complementation (Axiom ii)}
\textbf{Goal:} Take any set \(B \in f_*(\mathcal{A})\) and show that its complement, \(B^c = \Omega_2 \setminus B\), must also be in \(f_*(\mathcal{A})\).

\textbf{Reasoning:}
1. Start with the assumption: \(B \in f_*(\mathcal{A})\). By definition, this means \(f^{-1}(B) \in \mathcal{A}\).
2. State the goal for the complement: To show \(B^c \in f_*(\mathcal{A})\), we must show that its preimage, \(f^{-1}(B^c)\), is in \(\mathcal{A}\).
3. Connect the assumption and the goal: We can use a fundamental property of preimages which relates the preimage of a complement to the complement of a preimage \hyperlink{def:set-properties}{[3]}.

\textbf{Proof Step:}
Let \(B \in f_*(\mathcal{A})\). By definition, this implies that \(f^{-1}(B) \in \mathcal{A}\).
We want to show that \(B^c \in f_*(\mathcal{A})\). For this, we examine its preimage, \(f^{-1}(B^c)\). A key property of preimages is that the preimage of a complement is the complement of the preimage:
\[ f^{-1}(B^c) = \left(f^{-1}(B)\right)^c \]
We know \(f^{-1}(B)\) is a member of \(\mathcal{A}\). Since \(\mathcal{A}\) is a \(\sigma\)-algebra, it is closed under complements (Axiom ii). Therefore, the complement of \(f^{-1}(B)\), which is \(\left(f^{-1}(B)\right)^c\), must also be in \(\mathcal{A}\).
So, we have shown that \(f^{-1}(B^c) \in \mathcal{A}\). By the definition of \(f_*(\mathcal{A})\), this means that \(B^c \in f_*(\mathcal{A})\).

\vspace{1em}
\hrule
\vspace{1em}

\subsubsection*{Step 1.3: Checking Closure under Countable Unions (Axiom iii)}
\textbf{Goal:} Take any countable sequence of sets \(B_1, B_2, B_3, \dots\) from \(f_*(\mathcal{A})\) and show that their union, \(\bigcup_{j=1}^{\infty} B_j\), is also in \(f_*(\mathcal{A})\).

\textbf{Reasoning:} The logic is the same as before.
1. Start with the assumption: For every \(j \in \mathbb{N}\), we have \(B_j \in f_*(\mathcal{A})\). This means \(f^{-1}(B_j) \in \mathcal{A}\) for all \(j\).
2. State the goal for the union: To show \(\bigcup B_j \in f_*(\mathcal{A})\), we must show its preimage, \(f^{-1}(\bigcup B_j)\), is in \(\mathcal{A}\).
3. Connect them: Use the property that the preimage of a union is the union of the preimages \hyperlink{def:set-properties}{[3]}.

\textbf{Proof Step:}
Let \((B_j)_{j \in \mathbb{N}}\) be a sequence of sets such that \(B_j \in f_*(\mathcal{A})\) for all \(j\). By definition, this means \(f^{-1}(B_j) \in \mathcal{A}\) for each \(j\).
We want to show that the union \(B_{union} = \bigcup_{j=1}^{\infty} B_j\) is in \(f_*(\mathcal{A})\). We examine its preimage:
\[ f^{-1}(B_{union}) = f^{-1}\left(\bigcup_{j=1}^{\infty} B_j\right) \]
The preimage of a countable union is the countable union of the preimages:
\[ f^{-1}\left(\bigcup_{j=1}^{\infty} B_j\right) = \bigcup_{j=1}^{\infty} f^{-1}(B_j) \]
We started with the knowledge that each set \(f^{-1}(B_j)\) is in \(\mathcal{A}\). Since \(\mathcal{A}\) is a \(\sigma\)-algebra, it is closed under countable unions (Axiom iii). Therefore, the union \(\bigcup_{j=1}^{\infty} f^{-1}(B_j)\) must also be in \(\mathcal{A}\).
This means \(f^{-1}(B_{union}) \in \mathcal{A}\), which by definition implies that \(B_{union} = \bigcup_{j=1}^{\infty} B_j \in f_*(\mathcal{A})\).

\textbf{Conclusion for Part 1:} Since \(f_*(\mathcal{A})\) satisfies all three axioms, it is a \(\sigma\)-algebra over \(\Omega_2\). \qedsymbol

\newpage
\section{Part 2: Proving \(f^{-1}(\mathcal{B})\) is a \(\sigma\)-algebra on \(\Omega_1\)}

Now for the inverse image algebra, \(f^{-1}(\mathcal{B})\). This one is often called the \textbf{initial \(\sigma\)-algebra} or the \(\sigma\)-algebra \textbf{generated by \(f\)}. It's the smallest \(\sigma\)-algebra on \(\Omega_1\) that makes the function \(f\) measurable.
Let's recall its definition:
\[ f^{-1}(\mathcal{B}) = \{f^{-1}(B) \mid B \in \mathcal{B}\} \]
In plain words, this is the collection of all sets in \(\Omega_1\) that can be formed by taking the preimage of some set from \(\mathcal{B}\). We will now show this is a \(\sigma\)-algebra on \(\Omega_1\).

\vspace{1em}
\hrule
\vspace{1em}

\subsubsection*{Step 2.1: Checking for the Empty Set (Axiom i)}
\textbf{Goal:} Show that \(\emptyset \in f^{-1}(\mathcal{B})\).

\textbf{Reasoning:} To be in \(f^{-1}(\mathcal{B})\), the empty set must be the preimage of some set that is in \(\mathcal{B}\).

\textbf{Proof Step:}
We know that the preimage of the empty set is the empty set: \(f^{-1}(\emptyset) = \emptyset\).
We also know that \(\mathcal{B}\) is a \(\sigma\)-algebra on \(\Omega_2\), so by definition \(\emptyset \in \mathcal{B}\).
Therefore, \(\emptyset\) is the preimage of a set in \(\mathcal{B}\) (namely, \(\emptyset\) itself). This means, by definition, that \(\emptyset \in f^{-1}(\mathcal{B})\).

\vspace{1em}
\hrule
\vspace{1em}

\subsubsection*{Step 2.2: Checking Closure under Complementation (Axiom ii)}
\textbf{Goal:} Take any set \(A \in f^{-1}(\mathcal{B})\) and show that its complement, \(A^c = \Omega_1 \setminus A\), is also in \(f^{-1}(\mathcal{B})\).

\textbf{Reasoning:}
1. Start with the assumption: \(A \in f^{-1}(\mathcal{B})\). This means there exists some set \(B \in \mathcal{B}\) such that \(A = f^{-1}(B)\).
2. State the goal for the complement: To show \(A^c \in f^{-1}(\mathcal{B})\), we need to find some set \(B' \in \mathcal{B}\) such that \(A^c = f^{-1}(B')\).
3. Connect them: Let's express \(A^c\) in terms of preimages and see if we can find such a \(B'\).

\textbf{Proof Step:}
Let \(A \in f^{-1}(\mathcal{B})\). By definition, there exists a set \(B \in \mathcal{B}\) such that \(A = f^{-1}(B)\).
Now consider the complement of \(A\):
\[ A^c = \left(f^{-1}(B)\right)^c \]
Using the property of preimages \hyperlink{def:set-properties}{[3]}, this is equal to:
\[ A^c = f^{-1}(B^c) \]
Since \(B \in \mathcal{B}\) and \(\mathcal{B}\) is a \(\sigma\)-algebra, its complement \(B^c = \Omega_2 \setminus B\) must also be in \(\mathcal{B}\).
So we have found a set, namely \(B^c\), which is in \(\mathcal{B}\) and whose preimage is exactly \(A^c\). Therefore, by definition, \(A^c \in f^{-1}(\mathcal{B})\).

\vspace{1em}
\hrule
\vspace{1em}

\subsubsection*{Step 2.3: Checking Closure under Countable Unions (Axiom iii)}
\textbf{Goal:} Take a countable sequence of sets \(A_1, A_2, \dots\) from \(f^{-1}(\mathcal{B})\) and show that their union is also in \(f^{-1}(\mathcal{B})\).

\textbf{Reasoning:}
1. Start with the assumption: For each \(j \in \mathbb{N}\), \(A_j \in f^{-1}(\mathcal{B})\). This means for each \(j\), there exists a set \(B_j \in \mathcal{B}\) such that \(A_j = f^{-1}(B_j)\).
2. State the goal for the union: We want to show \(\bigcup A_j \in f^{-1}(\mathcal{B})\). This requires us to find a set \(B' \in \mathcal{B}\) such that \(\bigcup A_j = f^{-1}(B')\).
3. Connect them: Express the union of \(A_j\)s in terms of preimages.

\textbf{Proof Step:}
Let \((A_j)_{j \in \mathbb{N}}\) be a sequence of sets in \(f^{-1}(\mathcal{B})\). By definition, for each \(j\), there is a corresponding set \(B_j \in \mathcal{B}\) such that \(A_j = f^{-1}(B_j)\).
Consider the union of this sequence:
\[ \bigcup_{j=1}^{\infty} A_j = \bigcup_{j=1}^{\infty} f^{-1}(B_j) \]
Using the property that the union of preimages is the preimage of the union \hyperlink{def:set-properties}{[3]}:
\[ \bigcup_{j=1}^{\infty} f^{-1}(B_j) = f^{-1}\left(\bigcup_{j=1}^{\infty} B_j\right) \]
Since each \(B_j\) is in the \(\sigma\)-algebra \(\mathcal{B}\), their countable union \(\bigcup_{j=1}^{\infty} B_j\) must also be in \(\mathcal{B}\).
Let's call this union \(B_{union} = \bigcup_{j=1}^{\infty} B_j\). We have \(B_{union} \in \mathcal{B}\), and we have shown that \(\bigcup_{j=1}^{\infty} A_j = f^{-1}(B_{union})\).
Therefore, by definition, the union \(\bigcup_{j=1}^{\infty} A_j\) is in \(f^{-1}(\mathcal{B})\).

\textbf{Conclusion for Part 2:} Since \(f^{-1}(\mathcal{B})\) satisfies all three axioms, it is a \(\sigma\)-algebra over \(\Omega_1\). \qedsymbol

\section{Summary and Next Steps}

We successfully demonstrated that both the pushforward \(f_*(\mathcal{A})\) and the inverse image \(f^{-1}(\mathcal{B})\) are \(\sigma\)-algebras. The key to both proofs was not to construct the sets from scratch, but to use the definitions to translate the problem. We shifted the burden of proof back to the original sets \(\mathcal{A}\) and \(\mathcal{B}\), which we already knew were \(\sigma\)-algebras. The properties of preimages with respect to set operations (complements and unions) provided the necessary bridge.

\textbf{Why is this important?} The construction of \(f^{-1}(\mathcal{B})\) is especially crucial. It forms the basis for the definition of a \textbf{random variable}. According to Definition 1.45, a random variable is simply a \textit{measurable map}. A map \(X: \Omega \to \Omega'\) between two measurable spaces \((\Omega, \mathcal{A})\) and \((\Omega', \mathcal{A'})\) is measurable if the preimage of any measurable set in the codomain is a measurable set in the domain. In our notation, this is exactly the condition that \(X^{-1}(\mathcal{A'}) \subseteq \mathcal{A}\). This exercise shows that \(X^{-1}(\mathcal{A'})\) is a well-behaved collection of sets (a \(\sigma\)-algebra), which is a cornerstone for building the rest of probability theory on top of random variables.

\newpage
\appendix
\section{In-depth Explanations}

Here are more detailed explanations of the concepts we used in the proofs.

\subsection*{\hypertarget{def:sigma-algebra}{\textbf{[1] What is a \(\sigma\)-algebra?}}}
\emph{(Based on Definition 1.5 in the script)}

Think of a \(\sigma\)-algebra as a "club" of subsets of a larger space \(\Omega\). This club has very specific rules for membership to ensure it's robust enough for defining probabilities.
A collection of subsets \(\mathcal{A}\) of \(\Omega\) is a \(\sigma\)-algebra if it satisfies:
\begin{enumerate}
    \item \textbf{The club is not empty:} The empty set \(\emptyset\) must be a member. (Since members' complements must also be in the club, this automatically means the whole space \(\Omega = \emptyset^c\) is also a member).
    \item \textbf{Closure under complements:} If a set \(A\) is in the club, then everything in \(\Omega\) that is \textit{not} in \(A\) (its complement \(A^c\)) must also be in the club.
    \item \textbf{Closure under countable unions:} If you take any sequence of sets that are in the club (even infinitely many), their union (all elements combined) must also be a member of the club.
\end{enumerate}
These properties ensure that if we can measure the probability of some basic "events" (sets), we can also consistently measure the probability of their complements ("not A"), their unions ("A or B"), and their intersections ("A and B", which follows from the union and complement rules via de Morgan's laws).

\subsection*{\hypertarget{def:preimage}{\textbf{[2] What is a Preimage?}}}
Let's say you have a function \(f: \Omega_1 \to \Omega_2\). A function takes an element from the domain \(\Omega_1\) and maps it to a single element in the codomain \(\Omega_2\).

The \textbf{preimage} (or inverse image) works backwards. Instead of an element, it takes a whole \textit{subset} \(B \subseteq \Omega_2\) and asks: "Which elements in the domain \(\Omega_1\) get mapped into the set \(B\)?" The collection of all such elements is the preimage of \(B\).
\[ f^{-1}(B) = \{x \in \Omega_1 \mid f(x) \in B \} \]
\textbf{Example:} Let \(f: \mathbb{R} \to \mathbb{R}\) be \(f(x) = x^2\).
\begin{itemize}
    \item The preimage of the set \(B = \{4\}\) is \(f^{-1}(\{4\}) = \{-2, 2\}\).
    \item The preimage of the set \(C = [1, 9]\) is \(f^{-1}([1, 9]) = [-3, -1] \cup [1, 3]\).
    \item The preimage of the set \(D = \{-5\}\) is \(f^{-1}(\{-5\}) = \emptyset\), because no real number squares to -5.
\end{itemize}
\textbf{Important Note:} The notation \(f^{-1}\) here does not mean that the function \(f\) has an inverse function. The preimage is defined for any function, invertible or not.

\subsection*{\hypertarget{def:set-properties}{\textbf{[3] Properties of Preimages}}}
The proofs relied on two key properties that describe how preimages interact with set operations. For any function \(f: \Omega_1 \to \Omega_2\) and sets \(B_j \subseteq \Omega_2\):

\begin{enumerate}
    \item \textbf{Preimage of a complement:} \(\left(f^{-1}(B)\right)^c = f^{-1}(B^c)\) \\
    \textit{Why?} An element \(x \in \Omega_1\) is in the left side, \((f^{-1}(B))^c\), if \(x\) is NOT in the preimage of \(B\). This means \(f(x)\) is NOT in \(B\). But if \(f(x)\) is not in \(B\), it must be in \(B^c\). This means \(x\) is in the preimage of \(B^c\), which is the right side.

    \item \textbf{Preimage of a union:} \(\bigcup_{j} f^{-1}(B_j) = f^{-1}\left(\bigcup_{j} B_j\right)\) \\
    \textit{Why?} An element \(x \in \Omega_1\) is in the left side if it's in at least one of the sets \(f^{-1}(B_j)\). This means that for at least one \(j\), we have \(f(x) \in B_j\). But if \(f(x)\) is in at least one \(B_j\), it must be in their union, \(\bigcup B_j\). This means \(x\) is in the preimage of the union, which is the right side.
\end{enumerate}


\end{document}
