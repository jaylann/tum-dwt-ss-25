\documentclass[11pt,a4paper]{article}

% --- PACKAGES ---
\usepackage[a4paper, margin=2.5cm]{geometry}
\usepackage{amsmath, amssymb, amsthm}
\usepackage{graphicx}
\usepackage[utf8]{inputenc}
\usepackage{hyperref}

% --- HYPERREF SETUP ---
\hypersetup{
    colorlinks=true,
    linkcolor=blue,
    urlcolor=blue,
    citecolor=red
}

% --- THEOREM ENVIRONMENTS ---
\newtheorem{theorem}{Theorem}
\newtheorem{definition}{Definition}
\newtheorem{lemma}{Lemma}
\newtheorem{exercise}{Exercise}
\theoremstyle{definition}
\newtheorem*{solution}{Solution}

% --- DOCUMENT METADATA ---
\title{Exercise Walkthrough: Closure Properties of a $\sigma$-Algebra}
\author{Justin Lanfermann}
\date{25. June 2025}

\begin{document}

\maketitle

\begin{abstract}
    This document provides a detailed, step-by-step walkthrough for an exercise from the Discrete Probability Theory course. We will prove several key closure properties of a $\sigma$-algebra, relying only on its fundamental definition and De Morgan's laws. Each step is explained in detail, referencing the concepts as presented in the course script.
\end{abstract}

\section{The Exercise Statement}

\begin{exercise}
Suppose $\mathcal{A}$ is a $\sigma$-algebra over $\Omega \neq \emptyset$, let $A_0, A_1, A_2, \dots$ be a countable collection of sets in $\mathcal{A}$ and let $m \in \mathbb{N}$. Then the sets
\[
    \Omega, \quad A_0 \setminus A_1, \quad \bigcup_{i=0}^{m} A_i, \quad \bigcap_{i=0}^{m} A_i, \quad \bigcap_{i \in \mathbb{N}} A_i
\]
are also in $\mathcal{A}$. \\
\textit{[Hint: Use de Morgan’s law for the intersection properties.]}
\end{exercise}

\section{Foundational Concepts}

Before we begin, let's recall the core definition that our entire proof will be built upon. All our reasoning must trace back to these three fundamental properties.

\begin{definition}[$\sigma$-algebra \textnormal{\cite[Def. 1.5]{script}}]
\label{def:sigma_algebra}
A collection of subsets $\mathcal{A} \subseteq \mathcal{P}(\Omega)$ is called a \textbf{$\sigma$-algebra} on $\Omega$ if it satisfies the following three axioms:
\begin{enumerate}
    \item[(i)] $\emptyset \in \mathcal{A}$ (The empty set is in $\mathcal{A}$).
    \item[(ii)] For each $A \in \mathcal{A}$, we have $A^c \in \mathcal{A}$ (It is closed under complements).
    \item[(iii)] For each sequence $(A_i)_{i \in \mathbb{N}}$ in $\mathcal{A}$, we have $\bigcup_{i \in \mathbb{N}} A_i \in \mathcal{A}$ (It is closed under countable unions).
\end{enumerate}
\end{definition}

We will now prove each part of the exercise step-by-step. The numbering follows the solution provided in the exercise sheet.

\section{Step-by-Step Proof}

\begin{solution}
We are given that $\mathcal{A}$ is a $\sigma$-algebra and $(A_i)_{i \in \mathbb{N}}$ is a collection of sets, where each $A_i \in \mathcal{A}$.

\subsection*{Part (i): The whole set $\Omega$ is in $\mathcal{A}$}

\paragraph{Claim:} $\Omega \in \mathcal{A}$.

\paragraph{Reasoning:}
This property follows directly from the first two axioms of a $\sigma$-algebra.
\begin{enumerate}
    \item From axiom (i), we know that the empty set is in $\mathcal{A}$:
    \[ \emptyset \in \mathcal{A} \]
    \item From axiom (ii), we know that if a set is in $\mathcal{A}$, its complement must also be in $\mathcal{A}$. Applying this to the empty set, we get:
    \[ \emptyset^c \in \mathcal{A} \]
    \item The complement of the empty set, $\emptyset^c$, is the set of all elements in $\Omega$ that are not in $\emptyset$. By definition, this is the entire sample space $\Omega$.
    \[ \emptyset^c = \Omega \setminus \emptyset = \Omega \]
    \item Therefore, we conclude that $\Omega \in \mathcal{A}$.
\end{enumerate}

\subsection*{Part (ii): The difference of two sets is in $\mathcal{A}$}

\paragraph{Claim:} For any $A_0, A_1 \in \mathcal{A}$, the set difference $A_0 \setminus A_1$ is also in $\mathcal{A}$.

\paragraph{Reasoning:}
To prove this, we first express the set difference using operations we know $\mathcal{A}$ is closed under (or will prove it is closed under).
\begin{enumerate}
    \item The set difference $A_0 \setminus A_1$ can be written as the intersection of $A_0$ and the complement of $A_1$ (see \hyperlink{note:set_difference}{[3]} for an explanation):
    \[ A_0 \setminus A_1 = A_0 \cap A_1^c \]
    \item We are given that $A_0 \in \mathcal{A}$ and $A_1 \in \mathcal{A}$.
    \item By axiom (ii) of a $\sigma$-algebra, since $A_1 \in \mathcal{A}$, its complement $A_1^c$ must also be in $\mathcal{A}$.
    \item Now we need to show that the intersection of two sets in $\mathcal{A}$ (namely $A_0$ and $A_1^c$) is also in $\mathcal{A}$. This property, closure under finite intersection, is what we will prove in Part (iv). Assuming that result for a moment, we can conclude:
    \[ A_0 \cap A_1^c \in \mathcal{A} \]
    \item Thus, $A_0 \setminus A_1 \in \mathcal{A}$.
\end{enumerate}

\subsection*{Part (iii): The finite union of sets is in $\mathcal{A}$}

\paragraph{Claim:} For any finite $m$, the union $\bigcup_{i=0}^{m} A_i$ is in $\mathcal{A}$.

\paragraph{Reasoning:}
The axioms guarantee closure under \textit{countable} unions. A finite union is just a special case of a countable union. We can show this formally by constructing a countable sequence of sets from our finite collection.
\begin{enumerate}
    \item We are given a finite collection of sets $A_0, A_1, \dots, A_m$, all in $\mathcal{A}$.
    \item Let's define a new, \textit{infinite} sequence of sets $(B_i)_{i \in \mathbb{N}}$ as follows (see \hyperlink{note:padding}{[4]} for this technique):
    \[ B_i := \begin{cases} A_i & \text{if } i \le m \\ \emptyset & \text{if } i > m \end{cases} \]
    \item Is every set $B_i$ in our new sequence in $\mathcal{A}$? Yes. For $i \le m$, $B_i = A_i$, which is in $\mathcal{A}$ by our premise. For $i > m$, $B_i = \emptyset$, which is in $\mathcal{A}$ by axiom (i). So, $(B_i)_{i \in \mathbb{N}}$ is a countable collection of sets in $\mathcal{A}$.
    \item Now, we can take the countable union of this new sequence. Since taking the union with an empty set doesn't add any elements ($A \cup \emptyset = A$), this union is equivalent to our original finite union:
    \[ \bigcup_{i \in \mathbb{N}} B_i = A_0 \cup A_1 \cup \dots \cup A_m \cup \emptyset \cup \emptyset \cup \dots = \bigcup_{i=0}^{m} A_i \]
    \item By axiom (iii) of a $\sigma$-algebra, the countable union of sets in $\mathcal{A}$ must be in $\mathcal{A}$. Therefore:
    \[ \bigcup_{i \in \mathbb{N}} B_i \in \mathcal{A} \implies \bigcup_{i=0}^{m} A_i \in \mathcal{A} \]
\end{enumerate}

\subsection*{Part (iv): The finite intersection of sets is in $\mathcal{A}$}

\paragraph{Claim:} For any finite $m$, the intersection $\bigcap_{i=0}^{m} A_i$ is in $\mathcal{A}$.

\paragraph{Reasoning:}
This proof cleverly combines the closure under complements (axiom ii), the closure under finite unions (which we just proved in Part iii), and De Morgan's Law \cite[Lemma 1.2]{script}.
\begin{enumerate}
    \item We are given the sets $A_0, A_1, \dots, A_m$, all in $\mathcal{A}$.
    \item By axiom (ii), their complements are also in $\mathcal{A}$:
    \[ A_0^c, A_1^c, \dots, A_m^c \in \mathcal{A} \]
    \item From Part (iii), we know that $\mathcal{A}$ is closed under finite unions. So, the union of these complements must be in $\mathcal{A}$:
    \[ \bigcup_{i=0}^{m} A_i^c \in \mathcal{A} \]
    \item Now we use one of De Morgan's Laws (see \hyperlink{note:demorgan}{[2]}), which states that the complement of an intersection is the union of the complements:
    \[ \bigcap_{i=0}^{m} A_i = \left( \bigcup_{i=0}^{m} A_i^c \right)^c \]
    \item We already established that the set $\left( \bigcup_{i=0}^{m} A_i^c \right)$ is in $\mathcal{A}$. By axiom (ii), its complement must also be in $\mathcal{A}$.
    \item Therefore, $\left( \bigcup_{i=0}^{m} A_i^c \right)^c \in \mathcal{A}$, which means $\bigcap_{i=0}^{m} A_i \in \mathcal{A}$.
\end{enumerate}

\subsection*{Part (v): The countable intersection of sets is in $\mathcal{A}$}

\paragraph{Claim:} The countable intersection $\bigcap_{i \in \mathbb{N}} A_i$ is in $\mathcal{A}$.

\paragraph{Reasoning:}
The logic is identical to the finite case in Part (iv), but we use the axiom for countable unions directly instead of the derived property for finite unions.
\begin{enumerate}
    \item We are given a countable sequence of sets $(A_i)_{i \in \mathbb{N}}$, all in $\mathcal{A}$.
    \item By axiom (ii), their complements form another countable sequence of sets in $\mathcal{A}$:
    \[ (A_i^c)_{i \in \mathbb{N}} \text{ are all in } \mathcal{A} \]
    \item By axiom (iii), the countable union of these complements is in $\mathcal{A}$:
    \[ \bigcup_{i \in \mathbb{N}} A_i^c \in \mathcal{A} \]
    \item We apply De Morgan's Law for countable sets (see \hyperlink{note:demorgan}{[2]}):
    \[ \bigcap_{i \in \mathbb{N}} A_i = \left( \bigcup_{i \in \mathbb{N}} A_i^c \right)^c \]
    \item Since $\left( \bigcup_{i \in \mathbb{N}} A_i^c \right)$ is in $\mathcal{A}$, its complement must also be in $\mathcal{A}$ by axiom (ii).
    \item Therefore, $\bigcap_{i \in \mathbb{N}} A_i \in \mathcal{A}$.
\end{enumerate}
\end{solution}

\section{Summary}
By starting with the three basic axioms of a $\sigma$-algebra, we have successfully proven that it is also closed under several other important set operations:
\begin{itemize}
    \item Containing the whole space $\Omega$.
    \item Finite and countable intersections.
    \item Finite unions.
    \item Set differences.
\end{itemize}
These closure properties are what make $\sigma$-algebras a robust foundation for probability theory, ensuring that if we can measure the probability of some basic events, we can also measure the probability of more complex events constructed from them.

\newpage
\section{Additional Explanations}

\subsection*{[1] What is a $\sigma$-algebra?}
\label{note:sigma_algebra_full}
As stated in Definition \ref{def:sigma_algebra}, a $\sigma$-algebra $\mathcal{A}$ is a collection of subsets of $\Omega$ (called "events") for which we want to define probabilities. Think of it as the set of all "valid questions" we can ask about the outcome of a random experiment. For this collection to be mathematically sound, it must satisfy three rules:
\begin{enumerate}
    \item \textbf{The trivial event is included:} We must be able to ask about the probability of "nothing happening." So, the empty set $\emptyset$ must be in $\mathcal{A}$.
    \item \textbf{If you can ask about an event, you can ask about its opposite:} If we can ask "what is the probability of event A?", we must also be able to ask "what is the probability of event A \textit{not} happening?". This means if $A \in \mathcal{A}$, its complement $A^c$ must also be in $\mathcal{A}$.
    \item \textbf{If you can ask about a list of events, you can ask if at least one of them happens:} If we have a countable list of events $A_1, A_2, A_3, \dots$, we must be able to ask "what is the probability that at least one of these events happens?". This means the union $\bigcup_{i=1}^{\infty} A_i$ must also be in $\mathcal{A}$.
\end{enumerate}

\subsection*{[2] De Morgan's Laws}
\label{note:demorgan}
De Morgan's laws provide a crucial link between unions and intersections via complements. As stated in the script \cite[Lemma 1.2]{script}, for any collection of sets $\{A_i\}_{i \in I}$:
\begin{align*}
    \left( \bigcup_{i \in I} A_i \right)^c &= \bigcap_{i \in I} A_i^c \quad \text{(The complement of a union is the intersection of the complements)} \\
    \left( \bigcap_{i \in I} A_i \right)^c &= \bigcup_{i \in I} A_i^c \quad \text{(The complement of an intersection is the union of the complements)}
\end{align*}
In our proofs, we used the second law to turn an intersection problem into a union problem, which we could then handle with the $\sigma$-algebra axioms.

\subsection*{[3] Set Difference as an Intersection}
\label{note:set_difference}
The set difference $A \setminus B$ contains all elements that are in set $A$ but \textit{not} in set $B$. This can be stated as: "an element $x$ is in $A \setminus B$ if and only if ($x$ is in $A$) AND ($x$ is not in $B$)". The statement "$x$ is not in $B$" is equivalent to "$x$ is in the complement of $B$, $B^c$".
Therefore, we can rewrite the condition as: "($x$ is in $A$) AND ($x$ is in $B^c$)". This is precisely the definition of the intersection $A \cap B^c$.
\[ A \setminus B = \{ x \in \Omega \mid x \in A \text{ and } x \notin B \} = \{ x \in \Omega \mid x \in A \text{ and } x \in B^c \} = A \cap B^c \]

\subsection*{[4] The "Padding" Technique for Finite Collections}
\label{note:padding}
The axioms of a $\sigma$-algebra are defined for \textit{countable} (infinite) sequences of sets. To prove properties for \textit{finite} collections, we use a simple but powerful trick: we turn the finite collection into an infinite one in a way that doesn't change the result of the operation (union or intersection).
\begin{itemize}
    \item \textbf{For Unions:} We "pad" the sequence with empty sets ($\emptyset$). Since $A \cup \emptyset = A$, adding empty sets doesn't change the final union.
    \[ \bigcup_{i=0}^{m} A_i = A_0 \cup \dots \cup A_m \cup \emptyset \cup \emptyset \cup \dots = \bigcup_{i=0}^{\infty} B_i, \quad \text{where } B_i = \begin{cases} A_i & i \le m \\ \emptyset & i > m \end{cases} \]
    \item \textbf{For Intersections:} We "pad" the sequence with the whole space ($\Omega$). Since $A \cap \Omega = A$, adding the whole space doesn't change the final intersection.
    \[ \bigcap_{i=0}^{m} A_i = A_0 \cap \dots \cap A_m \cap \Omega \cap \Omega \cap \dots = \bigcap_{i=0}^{\infty} C_i, \quad \text{where } C_i = \begin{cases} A_i & i \le m \\ \Omega & i > m \end{cases} \]
\end{itemize}
This allows us to apply the axioms for countable collections to prove properties for finite ones.


\begin{thebibliography}{9}
    \bibitem{script}
    Niki Kilbertus.
    \textit{Discrete Probability Theory}.
    Summersemester 2025, Technical University of Munich.
\end{thebibliography}

\end{document}
