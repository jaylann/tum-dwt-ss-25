\documentclass[11pt,a4paper]{article}

% --- PACKAGES ---
\usepackage[a4paper, margin=2.5cm]{geometry} % Set page margins
\usepackage{amsmath, amssymb, amsthm} % Math packages
\usepackage[utf8]{inputenc} % Input encoding
\usepackage[T1]{fontenc} % Font encoding
\usepackage{hyperref} % For clickable links

% --- HYPERREF SETUP ---
% Nicer link colors
\hypersetup{
    colorlinks=true,
    linkcolor=blue,
    filecolor=magenta,
    urlcolor=cyan,
    citecolor=green,
    pdftitle={Exercise Walkthrough},
    pdfauthor={Justin Lanfermann},
}

% --- THEOREM ENVIRONMENTS ---
% For styling the exercise and proof
\newtheorem{theorem}{Exercise}
\newtheorem{lemma}{Claim}
\newenvironment{solution}{\begin{proof}[Solution]}{\end{proof}}

% --- DOCUMENT METADATA ---
\title{Exercise Walkthrough: Properties of Inverse Images}
\author{Justin Lanfermann}
\date{25. June 2025}

% --- BEGIN DOCUMENT ---
\begin{document}

\maketitle

\section*{Overview}

This document provides a step-by-step walkthrough of a fundamental exercise on the properties of inverse images. These properties are crucial in measure theory and probability. They form the logical foundation for the definition of a measurable function and, consequently, a random variable (as seen in \textbf{Definition 1.45} of the script). We will prove that the inverse image operation, $f^{-1}$, interacts predictably with basic set operations like subsets, unions, and complements. This ensures that structure is preserved when mapping from one space to another.

\begin{theorem}
Let $\Omega_1, \Omega_2$ be non-empty sets, $f : \Omega_1 \to \Omega_2$ an arbitrary mapping, and $\mathcal{C} \subseteq \mathcal{P}(\Omega_2)$ an arbitrary collection of subsets of $\Omega_2$. Then the following statements hold.
\begin{enumerate}
    \item[(i)] If $A, B \subseteq \Omega_2$ and $A \subseteq B$, then $f^{-1}(A) \subseteq f^{-1}(B)$.
    \item[(ii)] The inverse image of the union is equal to the union of the inverse images, meaning
    \[ f^{-1}\left(\bigcup_{A \in \mathcal{C}} A\right) = \bigcup_{A \in \mathcal{C}} f^{-1}(A). \]
    \item[(iii)] Given a subset $A \subseteq \Omega_2$, the inverse image of the complement is equal to the complement of the inverse image, meaning
    \[ f^{-1}(A^c) = (f^{-1}(A))^c. \]
\end{enumerate}
\end{theorem}

\hrulefill
\vspace{1cm}

\section*{Step-by-Step Solution}

We will tackle each part of the exercise individually, explaining the reasoning for each step.

% -------------------- PART (i) --------------------
\subsection*{(i) Monotonicity of the Inverse Image}

\begin{lemma}
If $A \subseteq B$, then $f^{-1}(A) \subseteq f^{-1}(B)$.
\end{lemma}

\begin{solution}
% I removed the problematic \paragraph command from here.
\begin{enumerate}
    \item \textbf{Goal:} We want to prove the set inclusion $f^{-1}(A) \subseteq f^{-1}(B)$.

    \item \textbf{Strategy:} To prove this, we will use the standard method for proving set inclusion \hyperlink{note2}{[2]}. We must show that any arbitrary element of the set on the left-hand side is also an element of the set on the right-hand side.

    \item Let $\omega$ be an arbitrary element in $f^{-1}(A)$. So, $\omega \in f^{-1}(A)$.

    \item By the definition of the inverse image \hyperlink{note1}{[1]}, if $\omega \in f^{-1}(A)$, then its image $f(\omega)$ must be an element of $A$. So, $f(\omega) \in A$.

    \item We are given the condition that $A \subseteq B$. By the definition of a subset, since $f(\omega) \in A$, it must also be true that $f(\omega) \in B$.

    \item Now, since $f(\omega) \in B$, we can again use the definition of the inverse image \hyperlink{note1}{[1]} to conclude that $\omega$ must be in the inverse image of $B$. So, $\omega \in f^{-1}(B)$.

    \item \textbf{Conclusion:} We started with an arbitrary element $\omega \in f^{-1}(A)$ and showed that it must also be in $f^{-1}(B)$. Therefore, we have proven that $f^{-1}(A) \subseteq f^{-1}(B)$.
\end{enumerate}
\end{solution}

% -------------------- PART (ii) --------------------
\subsection*{(ii) Preservation of Unions}

\begin{lemma}
$f^{-1}\left(\bigcup_{A \in \mathcal{C}} A\right) = \bigcup_{A \in \mathcal{C}} f^{-1}(A)$.
\end{lemma}

\begin{solution}
% I removed the problematic \paragraph command from here.
\begin{enumerate}
    \item \textbf{Goal:} We want to prove the set equality $f^{-1}\left(\bigcup_{A \in \mathcal{C}} A\right) = \bigcup_{A \in \mathcal{C}} f^{-1}(A)$.

    \item \textbf{Strategy:} To prove this, we use the standard method of double inclusion \hyperlink{note3}{[3]}. We will prove the inclusion in both directions.

    \item \textbf{Part 1: Show $f^{-1}\left(\bigcup_{A \in \mathcal{C}} A\right) \subseteq \bigcup_{A \in \mathcal{C}} f^{-1}(A)$.}
    \begin{itemize}
        \item Let $\omega \in f^{-1}\left(\bigcup_{A \in \mathcal{C}} A\right)$.
        \item By definition of the inverse image \hyperlink{note1}{[1]}, this means $f(\omega) \in \bigcup_{A \in \mathcal{C}} A$.
        \item By definition of a union of a collection of sets \hyperlink{note5}{[5]}, there must exist at least one set, let's call it $A'$, in the collection $\mathcal{C}$ such that $f(\omega) \in A'$.
        \item Since $f(\omega) \in A'$, the definition of the inverse image \hyperlink{note1}{[1]} tells us that $\omega \in f^{-1}(A')$.
        \item Since $\omega$ is in one of the sets of the collection $\{f^{-1}(A) \mid A \in \mathcal{C}\}$, it must also be in the union of this collection. Therefore, $\omega \in \bigcup_{A \in \mathcal{C}} f^{-1}(A)$.
    \end{itemize}

    \item \textbf{Part 2: Show $\bigcup_{A \in \mathcal{C}} f^{-1}(A) \subseteq f^{-1}\left(\bigcup_{A \in \mathcal{C}} A\right)$.}
    \begin{itemize}
        \item Let $\omega \in \bigcup_{A \in \mathcal{C}} f^{-1}(A)$.
        \item By definition of union \hyperlink{note5}{[5]}, this means there exists at least one set, let's call it $A''$, in the collection $\mathcal{C}$ such that $\omega \in f^{-1}(A'')$.
        \item By definition of the inverse image \hyperlink{note1}{[1]}, this implies that $f(\omega) \in A''$.
        \item Since $f(\omega)$ is in one of the sets of the collection $\mathcal{C}$, it must also be in the union of all sets in that collection. Therefore, $f(\omega) \in \bigcup_{A \in \mathcal{C}} A$.
        \item Finally, by definition of the inverse image \hyperlink{note1}{[1]}, this means that $\omega \in f^{-1}\left(\bigcup_{A \in \mathcal{C}} A\right)$.
    \end{itemize}

    \item \textbf{Conclusion:} Since we have shown inclusion in both directions, the two sets must be equal.
\end{enumerate}
\end{solution}

% -------------------- PART (iii) --------------------
\subsection*{(iii) Preservation of Complements}

\begin{lemma}
$f^{-1}(A^c) = (f^{-1}(A))^c$.
\end{lemma}

\begin{solution}
% I removed the problematic \paragraph command from here.
\begin{enumerate}
    \item \textbf{Goal:} We want to prove the set equality $f^{-1}(A^c) = (f^{-1}(A))^c$.

    \item \textbf{Strategy:} For this proof, we can use a more direct chain of logical equivalences ("if and only if", denoted by $\iff$). This is often more elegant than double inclusion when it's possible. An element $\omega$ is in the left set if and only if it is in the right set.

    \item Let $\omega$ be an arbitrary element in $\Omega_1$. Then:
    \begin{align*}
        \omega \in f^{-1}(A^c)
        &\iff f(\omega) \in A^c && \text{(by definition of inverse image \hyperlink{note1}{[1]})} \\
        &\iff f(\omega) \notin A && \text{(by definition of complement \hyperlink{note4}{[4]})} \\
        &\iff \omega \notin f^{-1}(A) && \text{(by definition of inverse image \hyperlink{note1}{[1]})} \\
        &\iff \omega \in (f^{-1}(A))^c && \text{(by definition of complement \hyperlink{note4}{[4]})}
    \end{align*}

    \item \textbf{Conclusion:} Since we have established a chain of equivalences from an element being in $f^{-1}(A^c)$ to it being in $(f^{-1}(A))^c$, the two sets must contain exactly the same elements and are therefore equal.
\end{enumerate}
\end{solution}

\newpage
\section*{Summary and Further Explanations}

\subsection*{Summary}

We have formally proven three key properties of the inverse image operation:
\begin{itemize}
    \item It preserves subset relations: $A \subseteq B \implies f^{-1}(A) \subseteq f^{-1}(B)$.
    \item It distributes over arbitrary unions: $f^{-1}(\cup A_i) = \cup f^{-1}(A_i)$.
    \item It commutes with the complement operation: $f^{-1}(A^c) = (f^{-1}(A))^c$.
\end{itemize}

These results are essential for probability theory. When we define a random variable $X$ as a measurable function from a probability space $(\Omega, \mathcal{A}, P)$ to a measurable space $(\Omega', \mathcal{A}')$, we require that for any measurable event $A' \in \mathcal{A}'$, its inverse image $X^{-1}(A')$ is also a measurable event in $\mathcal{A}$. The properties we just proved are exactly what you need to show that the collection of all such inverse images, $\{X^{-1}(A') \mid A' \in \mathcal{A}'\}$, itself forms a $\sigma$-algebra. This allows us to "pull back" the event structure from the output space to the original sample space, which is how we assign probabilities to outcomes of random variables.

\subsection*{Explanations of Key Concepts}

Here are more detailed explanations of the concepts referenced in the proofs.

\begin{description}
    \item[\hypertarget{note1}{[1] Inverse Image (Preimage):}] For a function $f: \Omega_1 \to \Omega_2$ and a subset $S \subseteq \Omega_2$, the inverse image (or preimage) of $S$ under $f$ is the set of all elements in the domain $\Omega_1$ that map into $S$. It is defined as:
    \[ f^{-1}(S) := \{\omega \in \Omega_1 \mid f(\omega) \in S\} \]
    Note that $f^{-1}$ here does not imply that $f$ has an inverse function; it is notation for an operation on sets. This is central to \textbf{Definition 1.45 (random variable)}.

    \item[\hypertarget{note2}{[2] Proving Set Inclusion ($\subseteq$):}] To prove that a set $X$ is a subset of a set $Y$, denoted $X \subseteq Y$, you must show that every element of $X$ is also an element of $Y$. The standard proof structure is:
    \begin{enumerate}
        \item "Let $x$ be an arbitrary element of $X$."
        \item Use definitions and given properties to show that $x$ must also be an element of $Y$.
        \item Conclude that since $x$ was arbitrary, the inclusion $X \subseteq Y$ holds.
    \end{enumerate}

    \item[\hypertarget{note3}{[3] Proving Set Equality (=):}] To prove that two sets, $X$ and $Y$, are equal, you must show they contain exactly the same elements. The most common method is \textbf{double inclusion}:
    \begin{enumerate}
        \item Prove $X \subseteq Y$.
        \item Prove $Y \subseteq X$.
    \end{enumerate}
    If both inclusions hold, it must be that $X = Y$.

    \item[\hypertarget{note4}{[4] Set Complement ($A^c$):}] Given a universe set $\Omega$ and a subset $A \subseteq \Omega$, the complement of $A$, denoted $A^c$, is the set of all elements in $\Omega$ that are not in $A$.
    \[ A^c := \Omega \setminus A = \{\omega \in \Omega \mid \omega \notin A\} \]
    In our exercise, for $A \subseteq \Omega_2$, $A^c = \Omega_2 \setminus A$, and for $f^{-1}(A) \subseteq \Omega_1$, $(f^{-1}(A))^c = \Omega_1 \setminus f^{-1}(A)$.

    \item[\hypertarget{note5}{[5] Arbitrary Union of Sets ($\cup$):}] For a collection of sets $\mathcal{C} = \{A_i \mid i \in I\}$, where $I$ is an index set, their union contains all elements that are in at least one of the sets in the collection.
    \[ \omega \in \bigcup_{i \in I} A_i \iff \exists i \in I \text{ such that } \omega \in A_i \]
\end{description}

\end{document}
