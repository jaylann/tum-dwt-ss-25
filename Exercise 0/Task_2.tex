\documentclass[11pt,a4paper]{article}

% --- PACKAGES ---
\usepackage[a4paper, margin=2.5cm]{geometry} % Set page margins
\usepackage{amsmath, amssymb, amsfonts}     % For advanced math typesetting
\usepackage[T1]{fontenc}                        % Font encoding
\usepackage{libertine}                          % A nice, clean font
\usepackage{hyperref}                           % For clickable links
\usepackage{xcolor}                             % For colors

% --- HYPERLINK SETUP ---
% Defines colors for the links
\hypersetup{
    colorlinks=true,
    linkcolor=blue!70!black,
    urlcolor=blue!70!black,
    citecolor=green!60!black,
    pdftitle={Exercise Walkthrough: Properties of Inverse Images},
    pdfauthor={Justin Lanfermann},
}

% --- CUSTOM COMMANDS ---
% This command creates a clickable reference number [x] that links to an explanation
\newcommand{\myref}[2]{\hyperlink{#1}{\color{red!80!black}[#2]}}

% --- DOCUMENT METADATA ---
\title{Exercise Walkthrough: Properties of Inverse Images under Mappings}
\author{Justin Lanfermann}
\date{25. June 2025}

\begin{document}

\maketitle

\section*{Overview}

This document provides a detailed walkthrough for an exercise on the properties of inverse images of a function. These properties are fundamental in measure theory and probability, especially for understanding the definition of a random variable (see \textbf{Definition 1.45} in the script). The exercise asks us to prove three key behaviors of the inverse image (or preimage) operation with respect to set-theoretic operations: subset inclusion, unions, and complements.

We will prove each statement using a standard technique called "element-chasing" \myref{ref:element_chasing}{2}, where we take an arbitrary element from one set and show it must belong to another.

\subsection*{Recap of Core Concepts}

Before we begin, let's quickly recall the most important definition for this exercise.

\paragraph{The Inverse Image (Preimage) \myref{ref:inverse_image}{1}}
Given a function $f: \Omega_1 \to \Omega_2$ and a subset $A \subseteq \Omega_2$, the \textbf{inverse image} of $A$ under $f$, denoted $f^{-1}(A)$, is the set of all elements in the domain $\Omega_1$ that map to an element in $A$. Formally:
\[
f^{-1}(A) := \{ \omega \in \Omega_1 \mid f(\omega) \in A \}
\]
It's crucial to remember that $f^{-1}$ here is \textit{not} an inverse function; it is an operation on \textit{subsets} of the codomain $\Omega_2$. The function $f$ does not need to be invertible.

\section{The Proofs in Detail}

Let's tackle each part of the exercise step-by-step.
The setup is: $\Omega_1, \Omega_2$ are non-empty sets, $f : \Omega_1 \to \Omega_2$ is a mapping, and $C \subseteq \mathcal{P}(\Omega_2)$ is a collection of subsets of $\Omega_2$.

\subsection*{(i) Monotonicity of the Inverse Image}

\textbf{Statement:} If $A, B \subseteq \Omega_2$ and $A \subseteq B$, then $f^{-1}(A) \subseteq f^{-1}(B)$.

\paragraph{Reasoning and Proof:}
This property is called monotonicity because the inverse image operation preserves the subset relationship. If a set $A$ is "smaller than or equal to" a set $B$, its preimage will also be "smaller than or equal to" the preimage of $B$.

To prove this, we will use the element-chasing method \myref{ref:element_chasing}{2}. We need to show that any element that belongs to $f^{-1}(A)$ must also belong to $f^{-1}(B)$.

\begin{enumerate}
    \item \textbf{Let $\omega$ be an arbitrary element of $f^{-1}(A)$.}
    Our goal is to show that this $\omega$ must also be in $f^{-1}(B)$.

    \item \textbf{Unpack the definition of $f^{-1}(A)$.}
    By the definition of the inverse image \myref{ref:inverse_image}{1}, if $\omega \in f^{-1}(A)$, it means that its image, $f(\omega)$, must be an element of $A$. So, $f(\omega) \in A$.

    \item \textbf{Use the given condition.}
    The exercise states that $A \subseteq B$. This means every element of $A$ is also an element of $B$.

    \item \textbf{Connect the steps.}
    Since we know $f(\omega) \in A$ (from step 2) and we know $A \subseteq B$ (from step 3), it logically follows that $f(\omega)$ must also be in $B$. So, $f(\omega) \in B$.

    \item \textbf{Repack the definition for $f^{-1}(B)$.}
    Now, if $f(\omega) \in B$, this is precisely the condition for $\omega$ to be an element of the inverse image of $B$. Therefore, $\omega \in f^{-1}(B)$.
\end{enumerate}
Since we started with an arbitrary element $\omega \in f^{-1}(A)$ and showed that it must also be in $f^{-1}(B)$, we have proven that $f^{-1}(A) \subseteq f^{-1}(B)$.
\hfill $\blacksquare$

\subsection*{(ii) Inverse Image of a Union}

\textbf{Statement:} The inverse image of the union is equal to the union of the inverse images:
\[ f^{-1}\left(\bigcup_{A \in C} A\right) = \bigcup_{A \in C} f^{-1}(A) \]

\paragraph{Reasoning and Proof:}
To prove this set equality, we must show inclusion in both directions:
\begin{enumerate}
    \item[a)] $f^{-1}\left(\bigcup_{A \in C} A\right) \subseteq \bigcup_{A \in C} f^{-1}(A)$
    \item[b)] $\bigcup_{A \in C} f^{-1}(A) \subseteq f^{-1}\left(\bigcup_{A \in C} A\right)$
\end{enumerate}

\textbf{Proof of a) ($\subseteq$):}
\begin{enumerate}
    \item Let $\omega$ be an arbitrary element in $f^{-1}\left(\bigcup_{A \in C} A\right)$.
    \item By definition, this means $f(\omega) \in \bigcup_{A \in C} A$.
    \item The definition of a union of sets tells us that if an element is in the union, it must be in at least one of the sets in the collection. So, there exists some set $A_0 \in C$ such that $f(\omega) \in A_0$.
    \item But if $f(\omega) \in A_0$, then by the definition of the inverse image, $\omega \in f^{-1}(A_0)$.
    \item Since $\omega$ belongs to at least one set in the collection $\{f^{-1}(A) \mid A \in C\}$, it must belong to their union. Therefore, $\omega \in \bigcup_{A \in C} f^{-1}(A)$.
\end{enumerate}

\textbf{Proof of b) ($\supseteq$):}
\begin{enumerate}
    \item Let $\omega$ be an arbitrary element in $\bigcup_{A \in C} f^{-1}(A)$.
    \item By definition of union, this means there exists some set $A_0 \in C$ such that $\omega \in f^{-1}(A_0)$.
    \item By definition of inverse image, this means $f(\omega) \in A_0$.
    \item Since $f(\omega)$ belongs to at least one set in the collection $C$ (namely $A_0$), it must belong to their union. Therefore, $f(\omega) \in \bigcup_{A \in C} A$.
    \item Finally, if $f(\omega)$ is in this union, then by definition, $\omega$ must be in the inverse image of this union. So, $\omega \in f^{-1}\left(\bigcup_{A \in C} A\right)$.
\end{enumerate}
Since we have shown inclusion in both directions, the equality is proven.
\hfill $\blacksquare$

\subsection*{(iii) Inverse Image of a Complement}

\textbf{Statement:} The inverse image of the complement is the complement of the inverse image:
\[ f^{-1}(A^c) = (f^{-1}(A))^c \]
Note that $A^c = \Omega_2 \setminus A$ and $(f^{-1}(A))^c = \Omega_1 \setminus f^{-1}(A)$.

\paragraph{Reasoning and Proof:}
This is another set equality, so we prove it by showing inclusion in both directions. This property is crucial for showing that preimages preserve the structure of a $\sigma$-algebra \myref{ref:sigma_algebra}{3}.

\textbf{Proof of a) ($\subseteq$):}
\begin{enumerate}
    \item Let $\omega \in f^{-1}(A^c)$.
    \item By definition, this means $f(\omega) \in A^c$.
    \item The definition of a complement means $f(\omega) \notin A$.
    \item If $f(\omega)$ is not in $A$, then $\omega$ cannot be in the set of elements that map to $A$. In other words, $\omega \notin f^{-1}(A)$.
    \item If $\omega \notin f^{-1}(A)$, it must, by definition, be in the complement of that set. So, $\omega \in (f^{-1}(A))^c$.
\end{enumerate}

\textbf{Proof of b) ($\supseteq$):}
\begin{enumerate}
    \item Let $\omega \in (f^{-1}(A))^c$.
    \item By definition, this means $\omega \notin f^{-1}(A)$.
    \item The statement $\omega \notin f^{-1}(A)$ means that the image of $\omega$ is not in $A$. That is, $f(\omega) \notin A$.
    \item If $f(\omega)$ is not in $A$, it must be in the complement of $A$. So, $f(\omega) \in A^c$.
    \item By definition of the inverse image, if $f(\omega) \in A^c$, then $\omega \in f^{-1}(A^c)$.
\end{enumerate}
Having shown both directions, the equality holds.
\hfill $\blacksquare$


\section*{Summary and Next Steps}

\paragraph{Key Takeaways}
We have formally proven three fundamental properties:
\begin{itemize}
    \item The inverse image operation $f^{-1}$ is \textbf{monotone} with respect to subset inclusion.
    \item $f^{-1}$ \textbf{distributes over unions} (it also distributes over intersections, as a good follow-up exercise).
    \item $f^{-1}$ \textbf{commutes with the complement} operation.
\end{itemize}
In simple terms, the inverse image operation $f^{-1}$ "plays very nicely" with all the basic set operations. This well-behaved nature is precisely why it is used to define the crucial concept of a random variable. A function $X: \Omega \to \mathbb{R}$ is a random variable if the inverse image of any "nice" set in $\mathbb{R}$ (a Borel set) is a "nice" set in $\Omega$ (an event in the $\sigma$-algebra $\mathcal{A}$). The properties we just proved ensure this structure is preserved \myref{ref:sigma_algebra}{3}.

\paragraph{Check Your Understanding}
The proof for intersections is very similar to the one for unions. Try to prove it yourself:
\begin{center}
    Show that \quad $f^{-1}\left(\bigcap_{A \in C} A\right) = \bigcap_{A \in C} f^{-1}(A)$.
\end{center}
(Hint: The logic is nearly identical, just replace "there exists" with "for all" where appropriate when dealing with the definitions of union and intersection.)

\paragraph{Further Reading}
It is very instructive to compare the behavior of the inverse image $f^{-1}(A)$ with the \textit{forward image}, defined as $f(S) := \{f(\omega) \mid \omega \in S\}$ for $S \subseteq \Omega_1$. You will find that the forward image does \textit{not} behave as nicely. For example, in general, $f(A \cap B) \neq f(A) \cap f(B)$. Thinking about a counterexample for this can really solidify your understanding of why the inverse image is so special.

\newpage
\section*{In-depth Explanations}
\begin{itemize}
    \item \hypertarget{ref:inverse_image}{\textbf{[1] The Inverse Image (Preimage):}}
    As stated, the inverse image $f^{-1}(A)$ is the set of all inputs that produce an output that lies within set $A$.
    \begin{itemize}
        \item \textbf{Analogy:} Think of $f$ as a recipe function that takes an ingredient $\omega \in \Omega_1$ and produces a dish $f(\omega) \in \Omega_2$. If $A$ is the set of all "spicy dishes," then $f^{-1}(A)$ is the set of all ingredients (like chili peppers, cayenne, etc.) that result in a spicy dish.
        \item \textbf{Concrete Example:} Let $f: \mathbb{R} \to \mathbb{R}$ be defined by $f(x) = x^2$. Let $A = [1, 4]$. The inverse image $f^{-1}(A)$ is the set of all $x$ such that $1 \le x^2 \le 4$. This corresponds to $x \in [-2, -1] \cup [1, 2]$. Notice that $f$ is not invertible as a function on $\mathbb{R}$, but the inverse image operation on its subsets is perfectly well-defined.
    \end{itemize}

    \item \hypertarget{ref:element_chasing}{\textbf{[2] Element-Chasing Proofs:}}
    This is the workhorse technique for proving relationships between sets. The logic is:
    \begin{itemize}
        \item To prove \textbf{subset inclusion} ($S \subseteq T$), you must show that any element of $S$ is also an element of $T$. The argument structure is: "Let $x$ be an arbitrary element in $S$. [...logical steps...] Therefore, $x$ must be in $T$."
        \item To prove \textbf{set equality} ($S = T$), you must show inclusion in both directions: $S \subseteq T$ and $T \subseteq S$. You do this by performing two separate element-chasing proofs.
    \end{itemize}
    This method breaks down abstract set statements into concrete logical steps about their elements.

    \item \hypertarget{ref:sigma_algebra}{\textbf{[3] Connection to $\sigma$-Algebras and Random Variables:}}
    From the script, a \textbf{$\sigma$-algebra} $\mathcal{A}$ on a set $\Omega$ is a collection of subsets of $\Omega$ that (i) contains $\Omega$, (ii) is closed under complements, and (iii) is closed under countable unions (\textbf{Definition 1.5}).

    A function $X: \Omega_1 \to \Omega_2$ is called a \textbf{random variable} (or more generally, a measurable map) if for every set $A_2$ in the $\sigma$-algebra on $\Omega_2$ (let's call it $\mathcal{A}_2$), its inverse image $X^{-1}(A_2)$ is in the $\sigma$-algebra on $\Omega_1$ (let's call it $\mathcal{A}_1$) (\textbf{Definition 1.45}).

    The properties you just proved are exactly the closure properties of a $\sigma$-algebra, but for the inverse image operation!
    \begin{itemize}
        \item $f^{-1}(\Omega_2) = \Omega_1 \in \mathcal{A}_1$ (Property (i) of $\mathcal{A}_1$)
        \item If $A_2 \in \mathcal{A}_2$, is $f^{-1}(A_2^c) \in \mathcal{A}_1$? Yes, because $f^{-1}(A_2^c) = (f^{-1}(A_2))^c$. If $f^{-1}(A_2) \in \mathcal{A}_1$, then its complement is too. (This is your proof (iii)).
        \item If $A_2, A_3, \dots \in \mathcal{A}_2$, is $f^{-1}(\bigcup A_i) \in \mathcal{A}_1$? Yes, because $f^{-1}(\bigcup A_i) = \bigcup f^{-1}(A_i)$. If all the $f^{-1}(A_i)$ are in $\mathcal{A}_1$, then so is their countable union. (This is your proof (ii)).
    \end{itemize}
    This shows that the inverse image operation perfectly preserves the structure of a $\sigma$-algebra. This is a profound and incredibly useful result.
\end{itemize}

\end{document}
