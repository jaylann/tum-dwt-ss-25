\documentclass[11pt,a4paper]{article}

% --- PACKAGES ---
\usepackage[a4paper, total={6in, 9in}]{geometry} % For setting page margins
\usepackage{amsmath, amssymb, amsthm} % For math symbols, theorems
\usepackage{xcolor}          % For colors
\usepackage[colorlinks=true, linkcolor=blue, urlcolor=blue, hypertexnames=false]{hyperref} % For clickable links

% --- DOCUMENT INFORMATION ---
\title{\textbf{Exercise Walkthrough: Intersection of $\sigma$-algebras}}
\author{Justin Lanfermann}
\date{June 25, 2025}

% --- THEOREM ENVIRONMENTS ---
% This makes theorem-like structures look nice and consistent.
\newtheoremstyle{tutorstyle}
  {\topsep}   % Space above
  {\topsep}   % Space below
  {\itshape}  % Body font
  {}          % Indent amount
  {\bfseries} % Theorem head font
  {.}         % Punctuation after theorem head
  {.5em}      % Space after theorem head
  {}          % Theorem head spec (can be left empty)

\theoremstyle{tutorstyle}
\newtheorem{exercise}{Exercise}
\newtheorem{theorem}{Theorem}
\newtheorem{definition}{Definition}
\newtheorem{remark}{Remark}

% --- CUSTOM COMMANDS ---
\newcommand{\powerset}{\mathcal{P}} % Power set symbol

\begin{document}

\maketitle

\begin{abstract}
    This document provides a detailed, step-by-step walkthrough of the proof that the intersection of any family of $\sigma$-algebras is also a $\sigma$-algebra. We will base our reasoning on the definitions and concepts introduced in the "Discrete Probability Theory" script by Niki Kilbertus. Each step is explained in plain language, aiming to build a solid understanding of this fundamental result.
\end{abstract}

\section{Introduction and Goal}

Hello! Today, we're going to tackle a foundational proof in probability theory. The exercise might seem a bit abstract at first, but the result is incredibly important. It's the theoretical bedrock that allows us to construct the specific $\sigma$-algebras we need, like the ones generated by a collection of sets.

Our goal is to prove that if you take any collection of $\sigma$-algebras on a given sample space $\Omega$, their intersection is also a $\sigma$-algebra. To do this, we first need to remember what a $\sigma$-algebra is.

\paragraph{Recall: What is a $\sigma$-algebra?}
From the script (\hyperlink{note1}{Definition 1.5, page 10 [1]}), a $\sigma$-algebra is a collection of subsets of $\Omega$ (called events) that satisfies three specific rules. It must contain the empty set, be closed under complementation, and be closed under countable unions. We'll use these three rules as a checklist for our proof.

\section{The Exercise Statement}

Let's formally state the problem we are going to solve.

\begin{exercise}
Let $\Omega \neq \emptyset$ be a set, $I$ be a nonempty \hyperlink{note2}{index set [2]} (finite, countable, or even uncountable), and let $\mathcal{A}_{\alpha}$ be a $\sigma$-algebra over $\Omega$ for each $\alpha \in I$. Then the set $\mathcal{A}$ defined as the \hyperlink{note3}{intersection [3]} of all these $\sigma$-algebras,
\[ \mathcal{A} := \bigcap_{\alpha \in I} \mathcal{A}_{\alpha} \]
is also a $\sigma$-algebra over $\Omega$.
\end{exercise}

\section{The Step-by-Step Proof}

\paragraph{Our Strategy:} To prove that $\mathcal{A}$ is a $\sigma$-algebra, we must show that it satisfies the three properties listed in \hyperlink{note1}{Definition 1.5 [1]}. We will check them one by one.

The core idea is simple: if something is in the intersection $\mathcal{A}$, it must be in \emph{every single} $\mathcal{A}_\alpha$. Since every $\mathcal{A}_\alpha$ is a well-behaved $\sigma$-algebra, we can use their properties to show that $\mathcal{A}$ must also be well-behaved.

\subsection*{Step 1: Checking for the Empty Set (Property i)}

The first rule for a $\sigma$-algebra is that it must contain the empty set, $\emptyset$.

\begin{itemize}
    \item \textbf{What we know:} By definition, each $\mathcal{A}_\alpha$ is a $\sigma$-algebra. Therefore, according to Property (i) of the definition, we know that $\emptyset \in \mathcal{A}_\alpha$ for every single $\alpha \in I$.

    \item \textbf{The logical step:} The intersection $\mathcal{A} = \bigcap_{\alpha \in I} \mathcal{A}_\alpha$ contains only the elements that are present in \emph{all} of the sets $\mathcal{A}_\alpha$. Since we've just established that $\emptyset$ is in every $\mathcal{A}_\alpha$, it must also be in their intersection.

    \item \textbf{Conclusion:} Therefore, $\emptyset \in \mathcal{A}$. The first property is satisfied.
\end{itemize}

\textit{(Note: The provided solution in the prompt mentions $\Omega$. This is also correct. If $\emptyset \in \mathcal{A}_\alpha$, then its complement $\Omega = \emptyset^c$ must also be in $\mathcal{A}_\alpha$ by property (ii). So $\Omega$ is in every $\mathcal{A}_\alpha$ and thus in their intersection. Starting with either $\emptyset$ or $\Omega$ works fine.)}

\subsection*{Step 2: Checking for Closure Under Complementation (Property ii)}

The second rule is: if a set $A$ is in the collection, its complement $A^c$ must also be in the collection.

\begin{itemize}
    \item \textbf{What we need to show:} For any arbitrary set $S \in \mathcal{A}$, we must prove that its complement $S^c$ is also in $\mathcal{A}$.

    \item \textbf{The logical step:}
        \begin{enumerate}
            \item Let's pick an arbitrary set $S$ such that $S \in \mathcal{A}$.
            \item By the definition of an intersection, if $S \in \mathcal{A}$, then $S$ must be an element of every single $\sigma$-algebra in the family: $S \in \mathcal{A}_\alpha$ for all $\alpha \in I$.
            \item Now, we know that each $\mathcal{A}_\alpha$ is a $\sigma$-algebra. This means each $\mathcal{A}_\alpha$ is closed under complementation (Property ii). So, if $S \in \mathcal{A}_\alpha$, then its complement $S^c$ must also be in $\mathcal{A}_\alpha$. This is true for all $\alpha \in I$.
            \item We have now shown that $S^c$ is an element of every $\mathcal{A}_\alpha$.
            \item Again, by the definition of intersection, if $S^c$ is in all of the sets $\mathcal{A}_\alpha$, it must be in their intersection, $\mathcal{A}$.
        \end{enumerate}

    \item \textbf{Conclusion:} We started with an arbitrary $S \in \mathcal{A}$ and showed that $S^c \in \mathcal{A}$. Thus, $\mathcal{A}$ is closed under complementation. The second property is satisfied.
\end{itemize}

\subsection*{Step 3: Checking for Closure Under Countable Unions (Property iii)}

The third and final rule is: if you have a countable sequence of sets that are all in the collection, their union must also be in the collection.

\begin{itemize}
    \item \textbf{What we need to show:} For any countable sequence of sets $S_1, S_2, S_3, \dots$ where each $S_i \in \mathcal{A}$, we must prove that their union $\bigcup_{i=1}^{\infty} S_i$ is also in $\mathcal{A}$.

    \item \textbf{The logical step:}
        \begin{enumerate}
            \item Let's take a countable sequence of sets $S_1, S_2, S_3, \dots$ where every set $S_i$ is in our intersection $\mathcal{A}$.
            \item By the definition of intersection, this means that the entire sequence of sets $\{S_i\}_{i \in \mathbb{N}}$ is present in \emph{each} $\mathcal{A}_\alpha$ for all $\alpha \in I$.
            \item We know that each $\mathcal{A}_\alpha$ is a $\sigma$-algebra, so it must be closed under countable unions (Property iii). Therefore, the union of our sequence, $\bigcup_{i=1}^{\infty} S_i$, must be an element of $\mathcal{A}_\alpha$ for every $\alpha \in I$.
            \item Since the union $\bigcup_{i=1}^{\infty} S_i$ is in every single $\mathcal{A}_\alpha$, it must, by definition, be in their intersection $\mathcal{A}$.
        \end{enumerate}

    \item \textbf{Conclusion:} We have shown that $\mathcal{A}$ is closed under countable unions. The third property is satisfied.
\end{itemize}

\section{Summary and a Glimpse Forward}

We have successfully checked all three properties from the definition of a $\sigma$-algebra.
\begin{enumerate}
    \item[\checkmark] $\emptyset \in \mathcal{A}$
    \item[\checkmark] For any $S \in \mathcal{A}$, we have $S^c \in \mathcal{A}$
    \item[\checkmark] For any countable sequence $S_1, S_2, \dots$ in $\mathcal{A}$, we have $\bigcup S_i \in \mathcal{A}$
\end{enumerate}
Since $\mathcal{A} = \bigcap_{\alpha \in I} \mathcal{A}_\alpha$ fulfills all the necessary conditions, we can confidently conclude that it is a $\sigma$-algebra.

\paragraph{Why does this matter?}
This result is not just a theoretical curiosity. It's the key that lets us define the concept of a \textbf{generated $\sigma$-algebra} (\hyperlink{note4}{Lemma 1.7, page 10 [4]}). Often, we start with a simple collection of subsets $\mathcal{E}$ (e.g., all open intervals on the real line) and we want to find the "smallest" $\sigma$-algebra that contains all of them. How do we know such a "smallest" one exists?

Our proof provides the answer! We can consider the (very large) family of \emph{all possible} $\sigma$-algebras that contain $\mathcal{E}$. We know this family is not empty because the power set $\powerset(\Omega)$ is always one such $\sigma$-algebra. By the theorem we just proved, the intersection of this entire family is also a $\sigma$-algebra. By its very construction, it is the smallest one containing $\mathcal{E}$. This is precisely the definition of the generated $\sigma$-algebra, $\sigma(\mathcal{E})$.

\newpage

\section*{Deeper Dive: Explanations}

Here are more detailed explanations of the concepts marked with clickable numbers.

\begin{description}
    \item[\hypertarget{note1}{[1] Definition 1.5: $\sigma$-algebra.}] A $\sigma$-algebra (or sigma-field) on a set $\Omega$ is a collection $\mathcal{A}$ of subsets of $\Omega$ that satisfies the following three axioms:
    \begin{itemize}
        \item[(i)] \textbf{Contains the empty set:} $\emptyset \in \mathcal{A}$. This represents the "impossible event".
        \item[(ii)] \textbf{Closed under complementation:} If a set $A$ is in $\mathcal{A}$, then its complement $A^c = \Omega \setminus A$ must also be in $\mathcal{A}$. (If an event can happen, the event "it doesn't happen" is also an event).
        \item[(iii)] \textbf{Closed under countable unions:} If you have a sequence of sets $A_1, A_2, A_3, \dots$ and all of them are in $\mathcal{A}$, then their union $\bigcup_{i=1}^{\infty} A_i$ must also be in $\mathcal{A}$. (If we can measure the probability of individual events, we can also measure the probability of "at least one of them happening").
    \end{itemize}
    This structure is essential because it defines the set of "measurable events"—all the questions to which we can assign a probability.

    \item[\hypertarget{note2}{[2] Index Set.}] An index set $I$ is simply a set used to label the elements of another collection. In this exercise, we have a collection of $\sigma$-algebras, and we use the elements $\alpha$ from the set $I$ to give each one a unique label, $\mathcal{A}_\alpha$. The exercise states that $I$ can be finite (e.g., $I=\{1, 2, 3\}$), countably infinite (e.g., $I=\mathbb{N}$), or even uncountably infinite (e.g., $I=\mathbb{R}$). Our proof works regardless of the size of $I$, which makes the result very powerful.

    \item[\hypertarget{note3}{[3] Intersection.}] The intersection of a family of sets, denoted $\bigcap_{\alpha \in I} \mathcal{A}_\alpha$, is the set of all elements that are members of \emph{every} set $\mathcal{A}_\alpha$ in the family. For an element $x$ to be in the intersection, it must satisfy $x \in \mathcal{A}_\alpha$ for all $\alpha \in I$. If it is missing from even one of them, it is not in the intersection.

    \item[\hypertarget{note4}{[4] Lemma 1.7: Generated $\sigma$-algebra.}] For any collection of subsets $\mathcal{E} \subseteq \powerset(\Omega)$, the $\sigma$-algebra generated by $\mathcal{E}$, denoted $\sigma(\mathcal{E})$, is defined as the intersection of all $\sigma$-algebras on $\Omega$ that contain $\mathcal{E}$. The proof we just completed guarantees that this intersection is indeed a $\sigma$-algebra. It is also, by construction, the \emph{smallest} $\sigma$-algebra containing $\mathcal{E}$. This is the standard way to build complex event spaces, like the Borel $\sigma$-algebra on $\mathbb{R}$, which is generated by the set of all open intervals.
\end{description}

\end{document}
